%% LyX 1.6.7 created this file.  For more info, see http://www.lyx.org/.
%% Do not edit unless you really know what you are doing.
\documentclass[10.5pt,english]{extbook}
\usepackage{charter}
\usepackage{avant}
\renewcommand{\ttdefault}{lmtt}
\renewcommand{\familydefault}{\rmdefault}
\usepackage[T1]{fontenc}
\usepackage[utf8]{inputenc}
\usepackage[paperwidth=5.06in,paperheight=7.81in]{geometry}
\geometry{verbose,tmargin=0.75in,bmargin=0.75in,lmargin=0.75in,rmargin=0.4in,headheight=0.25in,headsep=0.25in,footskip=0.4in}
\usepackage{fancyhdr}
\pagestyle{fancy}
\setcounter{secnumdepth}{-2}
\usepackage{babel}

\usepackage[unicode=true, 
 bookmarks=true,bookmarksnumbered=false,bookmarksopen=false,
 breaklinks=false,pdfborder={0 0 1},backref=false,colorlinks=false]
 {hyperref}
\hypersetup{pdftitle={The Woods},
 pdfauthor={Vasileios Kalampakas},
 pdfsubject={A science fiction fantasy novel}}
\usepackage{type1cm}
\renewcommand\normalsize{
   \@setfontsize\normalsize{10.5pt}{10.5pt}
   \abovedisplayskip 10\p@ \@plus2\p@ \@minus5\p@
   \abovedisplayshortskip \z@ \@plus3\p@
   \belowdisplayshortskip 6\p@ \@plus3\p@ \@minus3\p@
   \belowdisplayskip \abovedisplayskip
   \let\@listi\@listI}\normalsize  
\begin{document}

\title{The Woods}


\author{by Vasileios Kalampakas}
\maketitle
\begin{quote}
\vspace{3.5in}


\begin{flushright}
\textsl{}%
\parbox[c][1\totalheight][s]{0.55\columnwidth}{%
\begin{quote}
\textsl{with special thanks to my editor }

\textsl{who should have acted more like the City of Pyr}
\end{quote}
%
}
\par\end{flushright}
\end{quote}
\tableofcontents{}


\part{Ex Principia}


\chapter{Prologue}
\begin{quote}
It is by fortune alone that man maintains his bountiful existence,
unhindered by the forces beyond his grasp, unaware of what lies beyond.
Once that veil is lifted, who can foretell the future? 

-Hilderich D'Augnacy, \textsl{Visions of The Aftermath}
\end{quote}

\section{The hunter}

The hunter felt the air thicken. He ventured a look towards the gathering
storm. The wind started to gather into brief furious gusts, fallen
golden leaves dancing around him. Fleets of scurrying animals could
be heard, running for refuge, burrowing in whatever shelter the forest
could provide. 

His prey seemed indifferent to the coming weather though. The tracks
were heavy-laden, and steady, even purposeful. It wanted to be found.
It was as if it was challenging him. 

He crouched low amidst a thick cluster of bush and peered into the
ashen wall of birch trees, his senses on the edge, everything around
him pulsing with intensity, his mind focused on his prey. He emptied
his thoughts, and closed his eyes. He reached for the small pendant
around his neck, a golden tipped arrow felling a silver eagle. With
a silent prayer to God and a fire burning from within him, he darted
off into the trees, the shadows they cast over his hooded figure almost
failing to touch it. 

His feet carried him on a galloping pace, fast and steady, swiftly
cutting through the forest, following the clear, fresh tracks. The
scent he traced was still strange, neither like any animal or man
he had met or heard of before. He paused to sample the soil, feel
the ground with his hands. He tasted copper and a tingling sensation
lingered on his tongue. He strung an arrow through his bow and felt
exhilarated, brimming with a well-deserved confidence. The deeper
he went into the woods, the stronger the expectation for the kill
became. 

As his breathing became faster so did his pace, his form blending
in with the forest like a blur. His chest and leggings in brown and
dark green leathers, with the coppery scale mail overlaid made for
perfect camouflage in the autumn woods. Silent, swift and deadly,
he had the advantage. He had smeared his exposed flesh with fresh
dirt and his bodice was treated with boar fat, to preserve the leather
and cover up his human scent. 

As stealthy as possible, he followed a small stream that led to the
east, to the Hollows. The tracks started to become faint, the soft
dirt heavier and crisper. The scent grew mellow, the metal taste in
his mouth waned. He was losing his prey. 

He lay still for a moment and took a westward look, towards the clearing
from where he entered the forest. Above the forest ceiling he could
make out the grim blue and black of the stormy clouds. Lightning flashed
the scenery alight, the trees around him casting shadows like the
fanged mouth of a terrible behemoth, menacing and unescapable. The
trickle of the stream was drowned under an ominous thunder. The first
drops of rain fell on his cape. He felt he was now losing the advantage.
The scent of his prey would be washed away, his vision would be of
little help in the chaos of a storm, the forest would be filled with
distant shadows and all noise would be lost under the thunder and
the falling rain.

Without a hint of sight, sound or smell, his chest caught fire. The
hunter frantically searched for his foe, but then a second and third
fiery sting brought him to the ground. His face was contorted with
a soundless expression of pain and amazement. It felt like arrows
had pierced his armor but there were no arrow stalks or heads. It
mystified him, that he should die in such a fashion, never once seeing
what or who took his life. The world around him became grey and he
felt a rush of pain before his senses finally failed him. His breathing
stopped, and with the forest ceiling occupying his view, the hunter
exhaled laboriously and silently prayed to God for the last time.


\section{The dancer}

She revelled in the darkly lit chambers, her form that of a swirling
dervish, the locks of her hair mirroring the precious light with an
intoxicating sheen of honey and brown, ethereal smell of roses and
lavender pouring out of her skin. She moved as if the ground was a
mere illusion to be disregarded, her arms faintly bent upwards as
if praying, or caressing the lithe forms of young gods, and her face
had the impression of unborn awe, mesmerizing to see, inviolate to
the touch.

She danced to the sounds of incessantly beating drums in patterns
and rhythms deep and rumbling that thumped her very soul, following
a melody of strings as clear as a mountain spring erupting, a fresh
dew engulfing the chamber, and a band of flutes calling out to unseen
spirits, as if a ritual of old was being performed for her pleasure
alone.

The music reached a crescento, a groundshaking climax and she became
frenzied with passion, exhuming a mystical air of love, a beacon of
a haven for all the unloved ones, an unseen contract with a muse behind
each tempting gesture. Her faint gossamer dress swirled and failed
to contain her ethereal form in such a breathtaking way, that even
the flames of the brazers around the chamber flickered in tune with
her dancing form to cast shadows that seemed to have a life of their
own.

The crowd around her was silent and still, wearing almost identical
masks of brass, the few flames that illuminated the chamber adorning
them with golden hues of honey and a glimmer of sunlight. A single
man stood at the edge of the dancing stage, robed with heavy linen,
his face unmasked for everyone to see, tears running down his cheeks,
welling under his chin in an unwavering steady flow, his face a painful
mix of sorrow and awe, his eyelids closed shut in a vein attempt to
contain his tears.

At the climax of her dance, she laid her body down, planted her feet
and hands to the stage, her back forming an arc. And then she convulsed
in a familiar but unspoken way, the way of ecstasy, her pelvis moving
to the rhythm of the drums, faster and faster, as if an invisible
lover was being forced upon her. The music came abruptly to a stop
and utter silence filled the chamber. She sagged to her knees, her
hair conceiling her face completely. The silence was deafening, the
only sound her ragged breath. Then the unmasked man spoke while bowing
solemnly:

{}``Celia, I lack the words. The Chorus weeps in adoration. Let everyone
be witness to this moment: Celia danced the Edichoros, and the Gods
were pleased. So says the Chorus.''

In a transient moment of still time, the crowd of masks said in one
voice: {}``Aye''. As soon as the word was spoken, the masked men
dispersed as if answering to a silent summons and melted into the
shadows, as if they were never really there, as if they had been a
mirage, a background for this dance alone. The dancer and the unmasked
man still remained.

He extended his arms, palms facing upwards, a gesture to the dancer
or mayhap the Gods themselves. She stood up on her bare feet slowly,
her hands touching her thighs over her gossamer dress, strands of
her hair upon her bare shoulders. He spoke more softly, as if not
to be overheard, even though there was not a living soul around in
earshot.

{}``Celia, my love. Come.''

At his words, she touched his palms and drew closer to him. She looked
upon his face, wet with tears and lit by flickering flames, her hazel
eyes still glittering with ecstasy, alight with enthusiasm, and yet
forming a wizened look that belied her years.

{}``Amonas.'', she uttered his name with a feeling of relief.

{}``It is done. You need not worry anymore. Men and Gods alike will
remember this night for all time.'', Amonas said sweetly while gently
caressing her head.

{}``And you, will you cherish those tears?'', a faint smile forming
on her mouth, a playful expression and a gleeful look on her eyes.

{}``Need you ask?'', his eyes darting all over her features, to
her smooth hair, her sculpted nose, the lobes of her ears, her slender
neck, her measured lips and back to her stare.

{}``I am only a woman, Amonas. I have to.'', her neck craning to
meet his lips, as tall as he was.

{}``I'm not worthy of such a gift.'', said Amonas as he stood still,
black eyes peering through her closed eyelids.

{}``Speak no more.'', Celia said and hushed him by touching his
lips with hers, then embracing his neck with both hands, softly but
steadfastly guiding him towards her.

Afterwards they made love on that very stage, the silence of the chamber
broken only by the sound of sputtering candles and braziers.


\section{The Curator}

A man dressed in dark crimson robes and a sky blue sash made haste
up the keep's long winding staircase. Perspiration covered his craggy
old leathery face, his grey bearded chin still awash with the wine
he had spilt only moments earlier. 

\textquotedbl{}Not now, not here.Damn the fools, damn them!\textquotedbl{},
he kept repeating to the deaf, heavy-set stone walls, with almost
every breath. The flickering flame of the torch he held cast his form
in shadows over the stones of the stair's steps, on the dank walls.
The form of a stumbling, muttering old fool. Even the shadows seemed
to mock him, crouching even lower than him as the staircase finally
unwound onto the roof of the keep. 

\textquotedbl{}Why now, while I still draw breath?\textquotedbl{},
said the old man as he caught his breath, and started off to find
what it was he came looking for. As if long ago forgotten, he fumbled
around the roof, while a lukewarm dusk gave way to a chilling, gloomy
moonless night. 

He kept straightening his beard with one hand, while his eyes were
closed, and his other hand was raised, a pondering finger waving in
a hazy, uncertain rhythm. As if trying to catch up with a silent tune
only he could hear. Suddenly, he opened his eyes and set off with
his head intently searching the floor. 

The air smelled of liquorice and the burned wheatstalks of nearby
farmsteads. Planting season. He looked annoyed, trying to pick up
a loose cobblestone from the roof. It gave no purchase, and try as
he might he could not so remove it. He grumbled a mild curse, unbefitting
of his Office, status, demeanor, or personality. But to hell with
all that. It wouldn't matter soon, none of it would, he thought. 

Standing upright, he folded his arms and breathed deeply, his elbows
sagging slightly, his chin almost touching his chest. He sighed, and
then abruptly erupted with a flurry of curses, kicks, punches and
stomps again quite unseemly for a person of his stature. 

A Curator. By mutual assent among his peers, not a very prestigious
one, but nonetheless, a Curator. Forcing himself to calm down, he
drew a few deep breaths before standing over one ledge from the roof
and shouting, almost in a screech: 

\textquotedbl{}Hilderich!!Hilderich!\textquotedbl{} An answering shout
came from somewhere below: \textquotedbl{}Over here master Olom, over
here!\textquotedbl{} The curator leaned over the ledge, searching
for a face to direct his ire at, to no avail. He shouted once more,
throwing his fists wildly into the air.

{}``Hilderich, you mongrel! Fetch the keystone from my study, run
like there's no tomorrow!'', said the curator and Hilderich complied
smartly to the best of his abilities. A few moments passed then, and
a lot of things happened at once. First, a few steps behind where
the Curator stood, the air twisted and reality gave way to nothingness,
where a large kind of slit formed suddenly and the world seemed as
if made of a paper tapestry badly sewn together. The curator then
turned about slowly, and had time enough to yell one last time at
his pupil:''Run now! Find the one I dared not!''

As Hilderich ran outside the small keep, his gaze locked with the
despairing eyes of his master, whose last look implored him silently
to live. A hooded form seemed to grab Curator Olom from his neck with
a single armored hand and lift him over the keep's parapet.

Hilderic stood there for a moment transfixed, overtaken by the speed
and incredulity of what was taking place. When the hooded figure turned
its head slowly towards him, fear pierced his heart. When it threw
the Curator down to his death, it was still looking at him, only now
Hilderich was running, eyes wide with horror, hands gripping the keystone
with white knuckles.

The hooded figure stood where the curator had been standing only moments
earlier, the slit behind it now gone, as if it had been a trick of
the eye. With scarcely a thought, it jumped off the keep's roof and
landed on its two feet, barely registering the 60-foot drop. As it
turned to run after Hilderich, it paused instinctively after feeling
the bloodied hand of the dying Curator tugging at his plated boot.

{}``Had to make sure first.'', said the curator through agonising
pain and a broken jaw while clutching his pendant, when both him,
the keep, and the figure trying to smash his head with his boot instantly
became a thing of the past, bright white light suddenly filling the
dusky plains accompanied by an eerie, unnatural, unnerving silence. 

Hilderich only felt a haze of feat and a tingling at the back of his
head. He dared not stop or look back, he simply ran. And prayed.


\section{The jester}

The grand audience hall was fabulously lit through grandiose arched
windows on either side. Sunlight glistened off the brass and gold
etched everywhere around the hall, from chandeliers to decorative
ornaments, marked with the livery of the Castigator. Crests and banners
engraved with family mottos, finely crafted from materials of the
highest quality, hung in carefully positioned places around the hall,
denoting their respective family's status, lineage, and nobility.

Sweet aromas of burnt incense, cinnamon and musk permeated the air,
bouquets of freshly picked flowers from young maidens with the colors
of the rainbow were abundantly strewn around in neat vases and edifices
all around the thick marble pillars that supported the magnificently
painted dome, depicting wondrous scenes from the history, mythology,
and tradition of the Outer Territories.

A ruckus of tingling bells and a limerick tune of chords echoed in
the vastness of the audience hall, a single tinny voice singing along:
\begin{verse}
{\footnotesize Five pieces of gold that shone, and the sight of her
alone}{\footnotesize \par}

{\footnotesize Another man atop his throne, how will he ever atone}{\footnotesize \par}

{\footnotesize Bloody hands reach for the tome, will he ever dare
to come home?}{\footnotesize \par}

{\footnotesize This ballad may remind you of lore, I might even sound
a bore}{\footnotesize \par}

{\footnotesize The heart of it still remains, all will always be the
same}{\footnotesize \par}

{\footnotesize As long as clouds grace the sky, as long as He will
die}{\footnotesize \par}
\end{verse}
The jester played out the last part of the tune, merrily dancing around
the main hall, a wide smile seemingly permanently carved on his painted,
multicoloured face. As he hit the last of the chords, he ended his
performance with a wide curved bow towards the man who sat in the
center of the audience throne, his sole spectator, and waited there,
for a few moments, until he heard a morose voice:

{}``I tire of you too easily these days, Perconal. You used to be
more, ah.. Fun.'', said the voice that belonged to the Castigator
of the Outer Territories.

{}``I could do the leap-frog again, sire.'', the jester countered
with a hopeful proposition.

{}``That only seemed funny at the time since you leap-frogged onto
the Patriarch and the Procastinator Militant. Never saw a Procastinator
Militant fumble for his sword before.'', the Castigator responded
absent-mindedly, his head resting on his left hand, a goblet of wine
on his right one.

{}``No crowd today, sire. Who could I leap-frog onto then?'', the
jester insisted while fumbling with his crown of bells, his smile
turning into an ever more persistent grin.

{}``No crowd indeed. I believe I tire of crowds as well lately.''

{}``Perhaps .. an orgy?'', proferred the jester, shamelessly making
a rude pelvic thrust in the air, his hands mockingly grasping at an
imaginary waist.

The Castigator seemed to offer a little time to the idea, but a disapproving
nod of his head made the jester suddenly wear the face of a crying,
hurt man, shoulders slumped, hands knead together, as if pleading.

{}``No, Perconal. I'm not in the mood.''

{}``Then games sire! Games are always fun! And a challenge! Or are
you, afraid?Surely not!'', the jester said in a booming voice, and
then exploded into a series of mock athletic gestures, like running,
jumping and javelin throwing, flexing pitiful muscles, kneeling and
offering an invisible crown to the Castigator, looking as solemn and
expressionless as a grave.

{}``Games, you say?'', the Castigator seemed briefly intrigued,
and now rested his head on both hands, his voice slighty muffled.

While he seemed to ponder the idea, the jester scurried soundlessly
near the table where the goblet of wine lay, and with a wide grin
forming on his shallow face once again, he mischievously reached for
it. The Castigator took notice, but said nothing. Eyes darting to
and fro, the jester sipped some wine off the goblet, painted lips
smearing its bronze, delicately decorated surface, with white powder
and red and violet paint. As the jester closed his eyes and savoured
the exquisite vintage, he felt icy steel hard against his throat.

{}``Feuillout usually leaves too dry an aftertaste, don't you think?'',
the Castigator said to the jester, in all seriousness, his knife in
hand, set against the jester's throat, edge flashing bright from the
sunlight.

{}``Sire. I transgressed.'', replied the jester, any hint of grin
or smile cast out instantly.

{}``That you did, Perconal. I hate it when you do that. I thought
more of you.'', said the Castigator in an emphatically disappointed
tone of voice.

{}``I was tempted sire. I haven't tasted wine, any wine, since ..
I really can't remember. Truthfully.'', the jester almost cried out
the last few words, his head bowing in submission, his hands fumbling
with his chordus, careful not to touch any strings lest he make a
note.

{}``Well, no matter. Tomorrow you will be castigated, forty lashes
should be enough. People have been hanged for less. Water is so scarce,
yet you would risk your life to indulge in wine tasting, no less.
I think I'm growing a soft spot for you, Perconal.''

{}``Thank you sir. Can I at least have another sip, sire? It is so
sweet.'', said the jester with a half-formed smile and hint of a
gesture towards the goblet.

{}``Another sip? Ha! There you go Perconal, you actually made me
laugh. Ha ha!'', a loud laugh and a smile formed on the Castigator's
usually bored, flat face, before throwing the goblet on a nearby column,
wine spilling all over the black, shiny, green veined granite floor.

{}``There you go! Lap it up, you fool! Leave none for the maidens!'',
shouted the Castigator, a furious laughter welling up, unable to contain
it. And Perconal, the Jester, helplessly ran about the marble floor,
trying to sip as much of the spilled wine as he could, his bells and
jingles ringing and echoing in the empty hall.


\section{The boatman}

{}``'Tis none of my business, young sir, but given the chance and
all since I don't get many passengers through here this ferry, so
far away from the Basilica Road and all, might I ask where do you
come from? Beg your pardon.'', the boatman ventured in a fast talkative
manner, his sight affording his passenger a casual gaze, his hands
on the boat's rows beating them in and out of the water in a calm,
slow rhythm.

{}``Nicodemea. Up north. Is this safe? The fog, I mean.'', the young
passenger answered in an absent-minded fashion, his question trailing
with a hint of worry and nervousness, his eyes averted from the surrounding
fog and water, focused instead on the boat itself, as if an invisible
wall made such an effort mundane.

{}``Why shouldn't it be? The water's dead still, and there no rocks
on the other side, just green grass, young sir. You carry nothing
more than your person, so missing the platform shouldn't be a bother.
A simple matter, no worries, be there before you know it too. Looking
for a mule or a horse, any chance? Long way ahead, ain't I right?''

{}``But the fog. Isn't it ..'', the young man hesitated with a sour
face. 

{}``Thick? Damn thick fog this time of the year, lifts at around
noon, sets in before dusk. Pretty normal lad, come to think about
it, didn't catch your name now. Care to share in a friendly discussion?
Reilo's the name'', the ferry man interjected with a smile, part
glossy silver, part cavernous lack of teeth.

{}``Ahm, I'm Molo. Thessurdijan Molo.'', the young man said after
a small pause and some fidgeting with his cloak and belt before revealing
a gloved hand and proffering it to the boat handler.

{}``Can't right now lad, kinda caught up rowing, remember? But very
much obliged to meet you, young sir. I'm Reilo, Reilo the boatman.
Don't get many nice people like you around here.'', the ferry man
nodded in acknowledgement, underlining the fact he was rowing by enthusiastically
flapping the rows ineffectually above the water's surface, before
he added with a note of apprehension:

{}``Not to sound too promiscuous sir, but what's a nice gent like
you doing crossing these no-good-parts for?''

{}``Well you are quite talkative a fellow aren't you, Reilo? I'm
a student, on an errand, that's all.'', the young man rearranged
his cloak, and peered past the boat man, through the fog, without
success of glimpsing anything else than a grey oozing atmosphere and
a thin shiny sliver of murky water.

{}``Must be quite an errand to travel that far,eh?''

{}``That, it is indeed.'', said the young man sounding suddenly
grave. The fog started to lift about then, as a light breeze rushed
around them, the feeling of chilled clean air a welcome change on
their cheeks.

{}``There you are Molo sir, fog's lifting. Clockwork, eh?'', the
gaping mouth of the man lenting little of the associated perfection
to the word. 

{}``If you say so Reilo.''

{}``And once you're on the other side, how 'bout a getting some rest
for your aching feet, eh? I got a cousin, fine lad, comfy bed, real
straw and all, sensible prices mind you.'', the boat man pressed
on the advantage while rowing the last few yards towards the shore.

{}``I'm looking to keep on moving, thank you.'', Molo answered politely.

{}``Then a horse might come in handy? Got a nephew has a couple o'
fine workhorses, could sell you one off cheap if I put a word too.'',
Reilo blinked one eye in a way that offered one too many wrong connotations.

{}``I won't be needing any of that, thank you Reilo.'', said Molo,
stressing his expressed gratitude as well as his gentle patience by
accenting his thanks.

{}``Alright sir, hope no regrets later on.'', said the boatman,
somewhat disappointed his sales pitch didn't hit off as planned.

{}``Believe me, no regrets.'', answered Molo, and stepped off the
boat and onto the river's shore, his one hand on his knapsack, a walking
rod in the other one. He picked up a brisk pace and soon he met the
road going east. He checked his few belongings one last time as a
late afterthought and set off once again. 


\section{The Pilgrim}

His feet were sore. Cold air rushed to meet his face, the flimsy cloak
offering a little less than adequate protection. Tall grass grew on
either side of the rocky path through the hills. The cries of a crow
accompanied the howling gusts of the wind, the sky a bleak grey, just
like the days before. He looked around, searching for some kind of
shelter at least until the wind died down. He knew he had to rest
soon, his body ached and his legs felt like lead.

He spotted a large bark of a tree, a large oak, hollowed out, grizly
and old. He made a rush of straining effort to reach it quickly, further
up the hill. A little more pain and then he could sit for a change.
Perhaps even sleep, cold wind or not.

The oak was a perfect fit, large enough to lay down with only a small
opening a lean man had to go through sideways. He was lean. And hungry,
cold, tired and groggy. As he put down his knapsack, he grimaced with
pain from stiff muscles. He lay down, on the ground, and felt like
all his cares and troubles in the world were suddenly lifted, he felt
light as a feather, numbness encircling his senses.

He stretched his feet and looked up, a small crack on the bark letting
the sky seep through. His sight wandered to the clouds passing overhead,
grey tints on blue black matresses, like forlorn shapes running through
a twisting, foaming river. 

He closed his eyes, and muttered a prayer, thanking his God for the
timely shelter, feeling he was being looked after, cared for, like
his pilgrimage was an extraordinary thing, a matter of grave importance,
a mission he had to carry out, a mission worthy of every little help.
All he had to do was have faith, and he would persevere, and his God
would keep a watchful eye, and provide.

Soon, he fell asleep, laying there seemingly dead as the wood around
him. He dreamt, but he would remember nothing when he would wake up.
His chest rised and fell in a slow, laborious rhythm. And all around
the oak, the grass bend where the wind blew. And the crows had stopped
their crying, some of them perched on the very same oak.

A few drops of rain started falling, and pretty soon it turned into
a drizzle, thin and almost refreshing, a gift of shower the earth
accepted eagerly. He was dry, and he was warm. He thought to himself,
{}``God always provides'' and fell into a dreamless sleep.


\chapter{The City of Pyr}


\section{Dangers of the trade}

The first thing that assaulted Hilderich's senses were the smells.
Lost in a smelting crowd of city people, the smells were overpowering:
the acrid sweat of unkempt horses mingled with rosewood and cinder
scraps from the carpenters' workshops, heavy spices like cinammon
and uwe flared his nostrils while an essence of oils and meats wandered
through the air, the smell of filthy beggars waxing and waning around
every corner, its temporary absence filled in with incense from close
by temples and intoxicating perfumes from passing, illustrious carriages.

The mix of sounds though felt familiar, reminding him of bees buzzing
through the meadows back home, whole swarms feeding on the nectar
of roleva flowers, over a golden carpet swaying gently under the evening
gale. Now and then some voices stood taller than the rest, hints of
selling wares and the ever present and watchful Ministers announcing
laws, edicts and verdicts, punishments, and religious texts, all for
the good of the people. The cacophony was further accented by the
clacking sounds of hooves, pig cries and beggars' pleas.

Tall arches overhead cast angled shadows everywhere, the encircling
walls of the buildings like sheer cliffs towering over the palpitating
mass of people, animals and all the rest that could not readily fit
into either category. Blue-grey rock and lime mortar dominated the
market's landscape, the wear-torn cobblestones of the streets a hazy
washed white wherever the grit and mass of people allowed a small
glimpse. And street after street, wall after wall, bronze engraved
plaques embellished with holy texts and iconography hung on arches,
balconies and posts, in favor of the Gods, in memory and glory of
the Castigator and the Pantheon.

If the market was the heart of the city as its inhabitants claimed,
then what was said of Pyr being a heartless bitch seemed at once both
right and wrong. Every single cast, class and type of man was to be
found here, buying, selling, begging, stealing, killing, blackmailing
and dying. There were parts of the market where the suns had never
shown upon them since the city was built, dark corners where those
who entered usually did not reappear, and when they did, blood not
their own had been spilled.

The beggars had become part of the landscape, blinds, invalids, all
sorts of castaways and society's detritus, tugging away at embroidered
hems, pleading with sore voices and grotesque faces. Those of them
who bothered the wrong people time and again were soon beaten or stabbed
to death by lackeys and guardsmen, left for dead on the spot, attracting
the ire of honest, hard working merchants and artisans for not having
the decency to crawl away from near their workshops and stalls and
die someplace else where they would not put off potential customers.

But this was at the same time the place where everything of import
came to be; this was where produce from the surrounding fields and
indeed neighboring territories was gathered and sold to those who
could afford it. This was where artisans created common everyday wares,
materials and tools as well as delicate, commisioned works of art,
this was the place were deals and partnerships were entered and broken,
contracts signed and carried out, where everyone, whether a layman
or a noble, had some kind of business. 

This was the place were the Ministers' chants were heard every day,
preaching, teaching, and enforcing the religion that is Law. The market,
in that sense, was a living representation, a miniature of where and
how the people of Pyr lived their lives, and even how some of them
lost them.

Hilderich was drifting along the current of people flowing incessantly
through the market, occassionally bumping onto variously indifferent
or protesting men, taking care of the treads of carts, running heralds
and practicing pickpockets. He could hear the Minister from the next
street calling out a long list of names, and his eye caught two men
in an unlit alley cracking another man's skull, their shadowy outline
briskly contrasting with the lit background of the large street behind
the small alley. It seemed dishearteningly clear that this was business
as usual here, that some code of practice had been followed and the
formalities obeyed, killing a faceless man hidden away from the light
of the suns.

His almost random course took him closer to the Minister's spot, an
elaborate fountain made of granite, engraved with scenes of an historic
battle from the Heathen times he could remember learning about as
a child, but could not immediately recognise. The Minister held a
distaff on one hand, heavy-looking and oblique, and was still reading
names off a long unwinding scroll fitted in some kind of extendable
hook on the distaff that seemed purposefully designed. He wore a long
robe made of violet velvet with a gold embroidered hem and a silver-lined
crest of the Outer Territories on his chest, a small black cap with
a single emerald denoting his office, and both of the robe's arms
were filled with holy texts written in Lingua Helica, stitched in
purple silk. 

The ministers Hilderich had known made offerings to the Gods, upheld
the Law and taught it. These holy men seemed somewhat distasteful,
one might even call them pompous. He wondered what Master Olom's remark
would have been and he was reminded of the duty he had yet to fulfill.
He ignored the small mass of people gathered around the Minister as
well as the rest of the still unfolding list of names, and lost himself
once more in the throng of people, his senses acutely attuned by now,
searching for a sign that would bring him a step closer to finding
the one man he was searching for ever since that fateful night.

The thought of the word conjured in his mind a brief glimpse of that
night, an almost morbid recollection of what had happened. He had
barely had time to stop and ponder the minutiae, while trying to get
to Pyr as fast as possible, the place where he had to start his possibly
vain search. The more he thought about what had exactly happened,
the more he failed to grasp how everything had come crashing down
like an avalanche, one moment of perfect stillness followed by sheer
and utter terror, an unavoidable terrible fate. Such was the end of
Master Olom. 

Even though he would not admit to it, he felt he somehow cheated death
on that night, that it should have been him rather than his Master,
or at least he should have had the same luck, at least as a matter
of principle. Thankfully, wise Olom, though considered a relative
unknown, with a handful of friends and none in high places, thought
otherwise. 

Never pausing in his stride, Hilderich closed his eyes and clutched
the keystone his master had entrusted him with, ever so tightly. He
had taken extra precautions ever since he had to carry the strange
artifact with him. He always kept it on his person, and had fashioned
a small metal holder with a small but sturdy chain fastened to his
thick leather belt, resembling more the small cage of a sparrow, made
of thin sheets of metal he had scrounged off the stables of an inn,
the second day since he ran away from that explosion. 

That inexplicable explosion of light and heat, like hundreds of oil
barrels and steamers going off at once, only neither was to be found
even a day's ride away, much less unaccountably stored there at the
keep. It could be something Ancient, it must have been something Ancient.
But his knowledge of the Old People and their ways and artifacts was
still little, and as far as he knew, no such examples were in master
Olom's care, nor had he seen or heard any hint of such awesomely destructive
or powerful items during the last four years. But, what else could
it be? Unless the Gods had revealed themselves in blessed fury and
might. Wouldn't they have spared his master?

He let out a sigh without noticing, his head slightly slumped, his
expression sour. He had been so deeply entombed in thought, that when
he ventured a look he found himself utterly lost with no sense of
direction, in a part of the market conspicuously calm, lacking the
overwhelming mass of people that offered a false sense of safety that
was nevertheless more welcome than none at all.

The distant din of the market proper could still be heard, but he
had walked quite a distance and the crowd of people looked a little
more than a milling sea of garments and bustling feet. With the corner
of his eye he glimpsed a pair of shady figures was stalking him, probably
had been for a while, and were just about to gank up on him.

Hilderich was not a stoutly built fellow, and did not consider himself
neither a man of action or capable of putting up a serious fight.
But he had faith in his master, and his quest preceded everything
else. He had to preserve his life in order to preserve the keystone,
so he chose the most viable and logical course of action under the
circumstances: he ran like hell.

He suddenly darted off towards the direction of what seemed to be
a large bell tower, and ventured a slight look over his left shoulder
to get the bearing of the figures behind him. They were just beyond
hand's reach when he started running, his heavy cloak waving wildly,
feet scurrying on the cobbled street.

One of them cursed profoundly and the other one shouted at him to
stop, then both of them went on a chase after him. Hilderich went
right and left through alleys and larger streets, sometimes under
shadow and others under light, trying to keep the tower that somehow
seemed a safe place in sight, while at the same time trying to give
his pursuers the slip.

The sound of boots on stone was still unmistakeably behind him, and
sweat had started to pour out of his body. He went past small houses,
inns, and squares, while fleeing for what seemed his life, and curiously
enough his mind registered that not a housewive or elder man ventured
more than an indifferent look at the chase taking place in front of
their eyes, only the children paused in their play to look, point
and giggle excitedly.

His feet started to ache and his breathing became short and almost
painful, fire welling up in his lungs. He knew he could not keep this
up for long, and the tower he had set out to reach did not seem much
closer than earlier. He ventured a slight look over his right shoulder,
and couldn't see either of the figures chasing him. He listened intently
for a few moments and could not make out the distinct noise of chasing
boots, only rather his own two feet galloping achingly. He allowed
himself a drop in pace, easing his breathing, coming to a slow stop
near a shadowy wall, bent over with hands on his knees, throwing scared
looks around, hoping that the chase was over now.

{}``Don't ever run off like that again.'', the man's gruff voice
seemed to come through the wall of stone, but he was only hiding deeper
into the shadow of an adjoining alley, Hilderich instinctively turning
around, seeing a flash of light hinting at an unsheathed knife or
sword, its wholeness under the cover of dark.

For a moment he drew enough of a breath to dart off again in another
random direction, slipping away at the last minute like before, but
then he noticed the other man, a red-haired bearded brute, pieces
of armor showing underneath his shabby clothes, a jagged knife in
hand, standing a few paces to his right, steadfastly covering both
ways out of his predicament.

With no real options left, Hilderich suddenly leapt on the nearest
man, blade or not, in what could only really be a selfless last act
of defiance, thinking he was soon about to die, failing his master
for the last time. Nevertheless, he would give his all, and while
he leapt his gathered his fist aiming for the man's head, trying to
deliver as much pain as possible. The punch never connected.

The man still standing inside the shadows expertly and calmly took
a step back and bend his back slightly, with Hilderich's wild effort
going awry, hitting nothing but air, losing his balance and making
a counter-step. Just as he could feel his face freeze in astonishment,
he brought his other hand backwards in order to try and have another
go at punching his assailant. He did not manage that in any event
though, because he winced and doubled-up at the paralyzing pain from
the knee that had firmly and powerfully connected with his belly.

He felt his stomach empty and his sight go blurry, a hint of red clogging
his sight, his feet going limp and heavy, his breathing shallow. The
last thing he saw before he passed out, was a grinning mouth and the
icy clear flash of a steel blade.

\bigskip{}



\section{Nightveil}

She woke up with cold sweat on her forehead, her temples damp, locks
of her hair smack against her face. She drew the bedsheets against
her body, curling up onto the empty side of her bed. She felt the
unborn child inside her stir uncomfortably, as if awoken with terror
by the same dream. She laid her hands on her belly, gently caressing
it, feeling the child inside calm down, freely flowing in an inner,
warm sea of love and protection. 

The unborn fell silent, his tiny heart barely stating it was alive
and well inside his mother's womb. The mother on the other hand, was
still visibly shaken from her vision. She saw rivers of blood engulfing
her and her child, and fires bright as the sun all around her. She
saw Amonas' head on a pike, along with hundreds of other men's heads,
hellishly put on display in a gory show, seething masses of animals
rather than men cheering with voices that echoed like the very pits
of damnation.

The hair on her back was raised and she felt as helpless as a thawing
flake of snow, unable to comfort herself and lay back to sleep, the
shocking images of her sleep having burned through her waking heart
and mind. She stood up with some effort, and with slow attentive paces,
made it to the balcony. The night was invitingly chilly, offering
a crystal clear sky, the few clouds overhead like thin gossamer webs
the Gods might have woven to catch a falling star.

She peered out over the never-sleeping city, its market always alight
with torches and large common pyres, strange shadows flickering on
walls, fleetingly illuminated faces with blank expressions, as if
in a haze of ecstasy, whirling in the rhythmic dance of the city like
strands of cloth spun in the air.

The Ministry Tower and the Disciplinarium were lit as bright as day,
proud banners flung high and wide, a procession of ministers underway,
and another festivity in the Disciplinarium's garden that sported
tricks of fire and light was taking place, washing the rest of the
city with golden red hues and splashes of green and blue light, as
if there was not a care in the whole world.

The rugged brown stonework of the balcony felt rough under her smooth
touch. She moved her hand unconsciously across the stone while gazing
at the city, vainly searching for her loved one among the sands of
men. Her hand seemed to dance, ebbing and flowing to a melody even
she herself couldn't hear, like conducting an invisible chorus of
spirits of old inhabiting this very stone.

She closed her eyes momentarily and the image of Amonas' head haunted
her, his blood still pouring hot. She held her breath and moved her
lips in silent prayer, wishing her dream was nothing more than simple
fear of what lay ahead. Her faith should not waver, she must be strong
enough for Amonas' sake, and for their unborn child. Let the Gods
carry them forward and all will be as it should be.

Her child stirred once more, soothing her soul and bringing a faint
smile to her face. She went back inside, laid herself on her bed,
her hands resting on the still empty side, and slept in peace, her
face a statue of serenity. She dreamt no more that night.

\bigskip{}



\section{Inescapable Reality}

{}``He's coming around.''

{}``About time then.''

Hilderich knew he was still alive. The men's voices echoed faintly
in his throbbing head, pain stabbing him in the back of his neck,
muscles stiff from prolonged unconsciousness. He opened his eyes and
blinked furiously, his eyesight adjusting to a brightly lit chamber,
sunlight pouring from tall arched glass windows and a radiant dome
above. A small table occupied the middle of the room and a flimsy
looking cardboard adorned the opposite wall. The two men that had
attacked him were looking at him, the bearded brute standing up with
his hands in his pockets, the man with the gruff voice sitting down
at the table, a beam of light partly obscuring his face, dust motes
whirling silently in the air.

Hilderich lay in a small stone cot carved into a recess of a far wall.
The cot felt uncomfortable, hos body protesting slightly to his efforts
of movement. He flexed his arms and legs momentarily, and realised
he was not bound or restrained in any way. Puzzlement showed on his
face, and then the man who had bested him addressed him:

{}``You were out for almost a day. I am sorry we had to take somewhat
extreme measures, but we had to make sure you disappeared properly.
Giving us the run did not help, so we had to make it look like you
were being mugged. Hence the headache. I apologise for being rather
rough.'', the man's voice gruff but genuinely polite, almost friendly,
his words followed by a faint smile and condescending nod.

{}``Dunno if that put any sense innim though.'', the bearded man
grumbled under his breath audibly enough, regarding Hilderich with
a look akin to contempt.

{}``Philo..Please.'', said the man in the friendly voice, who appeared
to be the better of the brutish red-haired man.

Hilderich noted the brute's name was Philo, and watched him for moment
as he made a snorting sound and then crossed his hands across his
chest, leaned against a wall and quieted down, as the other man had
requested. The polite man continued:

{}``My friend here thinks you are somewhat dim and unforgivingly
naive for a place like Pyr. That remains to be seen. I think you will
prove very useful indeed.'', said the man, eyes level with Hilderich's
who was now sitting upright on the small cot.

{}``Who are you people?I thought you'd kill me.Why disappear? Why
am I here?'', the look on Hilderich's face was contorted, eyebrows
bent together, his puzzlement even more evident than before.

{}``You mean why did we not? Again, I apologise. My name is Amonas,
and this, as you know already, is Philo. Let's leave it at that for
now. There are things I can and things I cannot explain to you, at
least for the time being. You had to disappear because you were at
risk of being found by people with a different agenda. Once I make
myself clear, you will understand it was necessary for more significant
reasons, as well as for your protection. Soon enough, you will have
to reach a decision.''

{}``What kind of decision? Am I being threatened? You did not bind
my hands. Am I a prisoner?''

{}``You are in no way a prisoner. We just needed to talk to you in
safety. Please, hear what I have to say first. It is a matter of grave
importance, and it involves the keystone in your possession.'', Amonas'
tone carrying a hint of pleading and urgency, his words almost jarred
against each other.

{}``Aye, we knew.'', Philo added, a sever look cast upon Hilderich.

At the mention of the keystone, Hilderich was left wide-eyed. They
knew? Since when? Thoughts kept rushing through his mind, what he
should ask and what he might reveal that he should not. Was there
indeed real danger here? Were they telling him the truth? Was he being
safeguarded? He instinctively reached for the chain on his belt but
it was not there. They had taken the keystone. Had he failed already?
Was there a point to all this? His face took on an expression of silent
horror, mouth frantically opening and closing with no sound coming
out, like a fish out of the water, the spasms of death upon it.

Hilderich shouted in a trembling voice that could have been mistaken
for a shriek:

{}``Give it back! Thieves! You are common thieves!''

He was standing upright now, his head frantically turning from Amonas
to Philo and back, casting looks of urgent accusation, wide-eyed and
tense. He regained a measure of self-control, and clasped his hands
together and closing his eyes said in a clear, level voice:

{}``Please. Give it back.''

Amonas and Philo were exchanging dumbstruck looks, Hilderich's outburst
having caught them by complete surprise. When there was time enough
for them to answer, all they could do was break into a hearty fit
of laughter. Philo, while still laughing managed to speak:

{}``Thieves! And the.. `Please'? Listen to yourself, lad.''

Amonas cut in and took a beleaguered Hilderich by the arm, a calming,
reassuring voice issuing from his mouth:

{}``Fear not. We mean no ill. We consider you a friend, and if you'd
choose so, a brother. The keystone is safely with me, I have it on
my person. Here.'', and Amonas unclasped his cloak to reveal the
keystone and chain safely tucked away in a pocket on his leather bodice,
the chain safely attached to a ringed metal belt.

The sight of the keystone calmed Hilderich somewhat, but he still
protested, eyes darting around, a feeling of hurt in every look:

{}``What kind of friend hunts you down, knocks you out, and then
robs you of your most valuable posessions?''

{}``Your new ones do, at least the first time around. Please, listen.
We know about what happened to your master. It pains me as well, I
knew Olom personally.'', Amonas let that sink in a bit, recalling
pleasant memories, and then carried on:

{}``There is trouble brewing ahead in the Outer Territories. What
happened that night at your keep was not a singular or chance event.
The keystones of all Curatoriums are being gathered, in most cases
just handed away. And the timing is too perfect to be a mere coincidence.
We cannot yet ascertain exactly who is orchestrating this, but we
have a pretty good idea.'', Amonas glanced sideways at Philo, who
nodded in agreement.

{}``You knew my master? Do you have proof? Who is 'we'? Are you some
sort of group? Organisation? A cult?'', Hilderich did not like what
he was hearing, and the people in front of him were not making much
sense, what he heard felt very thin indeed.

{}``I knew him and loved him dearly. I'll show you proof, soon enough.
We are not a cult. Especially not a cult. You could call us a group.
We call ourselves Kin. Or Brotherhood of Old. Or Oldfolk. We come
by many names, some intentional, some unintentional. Some to help
us stay hidden, some to help us raise support. Yours, for instance.''

{}``You want my support? And this is how you go about asking people?
Ganking up on them in the middle of the street? Support for what?
Clubbing people in the head?!'', Hilderich's tone was incredulous
and he was about to go into a fit of hysterical laughter. Philo seemed
especially displeased with his tone and remarks, while Amonas stayed
calm and resolute, determined to make his point.

{}``We want your support in order to claim back our lands, overthrow
the tyranny of the Castigator and expose the hateful lie that is Pantheon.'',
Amonas words came out like the rush of an unstoppable river, ready
to wash away everything that might dare to stand in its path. He sounded
earthenly solid and unyielding, and Hilderich was too impressed to
actually register what he had just said.

{}``Well, that is.. Did you just.. But, that's heresy! And high treason!'',
Hilderich almost stammered the words aloud, unable to contain his
shock.

{}``Heresy is nothing but the leash that binds our blind brethren.
Treason is the act of paying tribute to false gods, letting our Kin
of Old fade like ghosts with nothing left to haunt..''

Amonas was in an instant transformed in Hilderich's eyes. What seemed
like a calm, reasonable person, yet mysterious to the point of incredulity,
had given its place to a fiery zealot of some misguided cult of heretics,
hot embers instead of eyes, a fanatic with an untenable goal.

Hilderich wasn't sure of what to say to such a man, but he settled
for what felt like the truth of things:

{}``You're crazy then? Deranged. Or just very misguided. But no.
You would have to be crazy for all that to run through your head without
bursting at the seams.''

{}``Am I Hilderich? Or have you been played like a fool? Like I was
before I saw truth, and righteousness. Like Philo was when he was
still a killer for hire? It is shocking, I admit it. But by the end
of the day, I promise you you'll see the truth of things as well.
If you place any value in it, you will be given sight and hearing
anew, and you'll be ready to decide for yourself. Become brothers,
or stay just friends.''

{}``You seem too confident in what you say, and maybe you are a special
breed of crazy. But I cannot even begin to imagine how you can prove
that these aren't the rumblings of a madman, that Gods forbid, our
religion is a lie. I might confess, I find some of the Castigator's
and the ministers decisions and punishments harsh and unbefitting,
but I dare not judge their wisdom, or the Law as we have been taught
it. It is one thing to doubt the people that uphold the Law, and quite
another to judge what is Holy Law, inviolate and heaven-sent.''

Hilderich was now exuding an air of authenticity and oratory skill,
dressed with a rarely exhibited confidence, resembling an almost staunch
belief. He seemed like he knew he was right, and Amonas wrong.

Amonas grinned, nodding his head to Hilderich, then looking up to
meet his almost defiant gaze, and telling him in his gruff, steady
voice:

{}``The seed of doubt is already sown in your soul, Hilderich. Tonight
it shall spring to bear fruit at last. It is a good sign, for I believed
you would be too unwilling to see the truth, but this may yet prove
not to be the case.''

{}``I still think you are madmen. Suppose it is true, the Law and
the Gods are a lie? What then? What is the truth? And if you, for
the case of argument, succeed in your purpose? What replaces everything
we live by every day? Are you going to just blink and overthrow the
Castigator, and the army and all the nobles who will stand against
you? Or do you cherish a fantasy where all but a few will join your
cause, and no blood will be spilled in your day of triumph?''

{}``You are more of a thinker than I had believed, Hilderich. Above
all, a man of logic. Your own questions will be answered in due time.
A bit of patience will go a long way. And in any case, we can't allow
you to leave just yet, not before you witness what will invariably
change your opinion, for better or worse. You will come to see that
we are not crazy, and that we know that chaos and mayhem will erupt
once we decide the time has come to act. I have no fantasies and will
bear no misgivings: blood will be spilled inevitably, but not in vain.
These are not the endeavours of bloodthirsty warmongers or power-hungry
dissidents, Hilderich. We are hope incarnate.''

{}``If you say so, it must be true, then.'', Hilderich's tone mocking
beyond the point of insult. That remark made Philo move and fidget
with suppressed anger, pleading looks of putting the audacious little
man in his place by way of mild violence. Amonas would have none of
that and dismissed Philo's silent protests with an outstretched hand,
replying in a serious and honest tone of voice:

{}``I can only force you to remain until that promise is delivered.
Then you can freely go your own way, but the keystone will be kept
with us.''

{}``So not much of an offer, but rather a blackmail. I can go, but
the keystone stays? And you ask me to trust in you under these conditions?
Would you think differently if you were in my spot?''

{}``I have been in a similar spot, and no, I did not think differently.
And that is why I believe you'll be our brother by the first light
of dawn tomorrow.'', Amonas smiled heartily for Hilderich to see,
but his expression remained distant and withdrawn.

{}``You place too much on belief and court with arrogance, Amonas.
I don't like all of this one bit. But I seem to be at your mercy,
as a matter of fact.''

{}``You will prove more valuable and insightful than I had imagined,
Hilderich. It is exactly that false belief and that vaunted arrogance
that we seek to expose, and since you place no value in those, I consider
you a brother already, a man free of those poisons.''

{}``We shall see, but don't bet on it.''

Hilderich sat down on the cot once again, hands on his knees, seemingly
resigned to his fate but showing a somewhat proud stand over his beliefs,
his body stance and facial expression emitting a message that he would
not succumb to such underhanded tactics, if not in essence, as a matter
of principle.

{}``I am not a gambling man Hilderich, you need not worry. There
is still some time ahead of us. You must be famished. Do you wish
to eat now? I know we have been less than welcoming, but the circumstances
got the better of us. Please, indulge us.''

Hilderich pondered this for a while and in his mind dismissed the
idea the food would be poisoned, since if they wanted him dead, they
wouldn't have bothered with all this. It seemed somewhat of a concession
to the man effectively holding him captive, but the inescapable reality
of not having eaten for more than a whole day struck home in the end.
He nodded in acknowledgement and then added verbally: 

{}``That would be nice.''

Amonas smiled politely and gestured to Philo to get some food and
water for the three of them, and promptly Philo vanished behind a
rather small wooden door, his steps faintly audible as he seemed to
traverse a staircase.

{}``I'm told our most hospitable friends here have prepared a delicacy
today: lamb stew with uwe and knop leaves. Should be rather tasty.''

{}``I still think you are crazy, Amonas. Just a very curious, strange
kind of crazy person, apparently with a culinary taste to match.''

{}``Hilderich, I confess, I have a soft spot for uwe. It's an acquired
taste.''

Hilderich just shook his head in disbelief, a small sigh escaping
his lips. Amonas smiled ever more broadly at that particular reaction,
and pretty soon the strong smell of boiled uwe wafted through, hinting
at Philo's return. Amonas seemed rather expectant, and could not help
asking Hilderich:

{}``If you don't like uwe I believe a different arrangement can be
made, something like ham or breadpie.''

{}``Uwe lamb stew will be fine. I can manage.''

Hilderich caught himself being actually irritated at Amonas' inexplicable
insistence on food, which he found rather childish, further enhancing
his opinion that the man was, indeed, deranged if not completely crazy.

{}``Oh well, I had to try and steal that extra serving.''

Philo returned and entered through the door with some difficulty due
to the fact he was trying to balance two plates of stew on one hand
and arm and a large bowl on the other hand. He precariously managed
to reach the safety of a nearby table, putting the bowl down first,
and then unloading the plate off his arm. Seeing that noone cared
to assist him he threw around a few looks of mixed hurt and mild anger,
before pulling a chair and sitting at the table, then proclaiming:

{}``If you gonna join the table be quick about it.''

Amonas sat excitedly in front of the bowl of stew, and Hilderich guessed
the other plate was laid out for him and sat accordingly, but couldn't
help noticing and asking:

{}``Shouldn't there be another plate for Amonas?''

Amonas had already dug in with a spoon, and seemed quite indifferent
at anything else that went on around him and rather focused at enjoying
his substantially rich meal.

{}``No point in using a plate if he's gonna eat a whole bowl. So
I brought the bowl. Now, eat.'', Philo said with a hint of retired
disapproval.

Hilderich nodded slowly, shrugged and dug in like the rest. He took
a careful taste, and then munched and gulped eagerly. It tasted delicious
indeed.

\bigskip{}



\section{Darkly lit night}

The Disciplinarium's large audience hall was exquisitely decorated
with fine tapestries, hung from the columns and walls with golden
ropes and aggrandized with silk laces, freshly picked fragrant flowers
and all manners of highly luxurious pomp and ceremony. Though night
had already fallen, giant ornate silver and brass chandeliers hung
from the high ceilings, illuminating the grand hall with the light
of thousands candles, their beams of light enhanced and mirrored by
the all the brass, gold and silver decorations strewn around almost
every object in the hall, making them glitter and shine, magnifying
their splendor tenfold.

Delicately detailed lifelike oil paintings adorned each wall, previous
Castigators and Archministers, Procastinator Militants and Patriarchs,
noble supporter families of the Castigator, every last important person
that was notably recorded in history books was to be found here in
the form of awe inspiring portraits, paintings and sculptures from
the most talented artists of each generation from around the lands.
The mass of people was still flowing slowly but steadily into the
hall, and for a time it would seem like the small swarm of men was
going to swell to inelegant numbers. But the Castigator's people who
were in charge of the eventful night, those who orchestrated and planned
who was to be given the praise of summons, also decided the place
and time of his appearance as well as whether or not he should have
the privilege of being able to dine at the same table as the Castigator,
albeit always at an innocuous distance, a table seat that implied
an immense increase of status and almost unrivalled political power.

As the time for the opening ceremonies for the grand festive night
grew closer, all needed preparations were being doubly checked, and
the gathering of guests efficiently monitored. Everything seemed to
be in place, refreshments and drinks served in silver plated cups,
and sweetmeats, fruit, and fine pastries circulating among the crowd
in golden platters by busy servants, dressed in fine cloth wearing
the green livery of the Castigator's office, a white eagle bearing
a book and a key, a snake held in its beak. 

People were chatting in low voices, politely exchanging greetings
and news when being personal acquaintances, though some of the more
brazen guests that either lacked the knowledge of etiquette or were
in a position to ignore it as a whole or in part, were already laughing
heartily at jokes or anecdotes between friends and close acquaintances.

Everyone attending had been careful to dress as stylishly as possible,
and according to wealth and status, there were examples of extravagant
overdressing, with some people resembling moving heaps of a treasury
or jewelry shop. Others preferred to overstate their presence with
exotic cloths and tailorings, usually uncommon and outlandish, suggesting
time and money spent just for this one occasion. Indeed, everyone
was wearing the best and brightest they could afford, and maybe some
had even took on a loan to have something special tailor-made in order
to try and stand out in the crowd, in a bid to improve their fame
and fortune.

The atmosphere in the hall was generally convivial though mildly restrained
because of the premises and significance of the night. The festivities
were taking place in order to commemorate the Castigator's 25th Term
in Office, which coincided with the Pacification of Zaelin, the last
of the Territories to be enlightened and brought under the Law of
the Pantheon. Rumors circulated among the nobles elite and people
in the army and the Ministry that the Castigator would be announcing
a decision of major importance that would stir up the relatively still
affairs of the Territories, perhaps ushering a new era of glory to
the Gods.

In any case, it would be an eventful night, with dancing troupes of
wondrous abilities, unsurpassed technique and airy grace performing
for the duration, bards of worldly renown and enchanting voices singing
the Mythos in praise of the Castigator and the Pantheon, a telling
re-enactment of the Pacification of Zaelin and the striking down of
the last Heathen, Parnoth Larthiel, the Last Ignorant. The re-enactment
would also offer a kind of prize to one of the guests who would be
lucky enough for his name to be drawn amongst hundreds: he would have
the chance to play as the Castigator in the final duel with Parnoth,
the unclaimed role filled in by a lawbreaker due for execution. The
honoured guest in killing the lawbreaker-Parnoth would be spilling
heretic blood in the service of the Ministry, the Castigator and the
Law, possibly the highest service to the Pantheon possible, save sacrificing
one's self while enforcing or upholding the Law.

A huge oblong platinum chime resounded by the stroke of an ornamental
ram swung by two Protectors, the Castigator's personally hand-picked
guard, and judging by their imposing physical builts apparently chosen
chiefly for their brawn. The sound of the chime drowned out the chatter
of the milling guests, reverberating with a majestic effect in the
audience hall, and signalling the official commencement of the festivities.
Acoustics was one of many things not left to chance when the grandiose
chamber was built.

The crowd of guests went silent and ushered by dutiful servants and
thick-set expressionless Protectors, gathered on two opposing sides
of the hall, leaving a wide stretch of room where the Castigator would
soon walk on. Indeed, the voice of the Chief Functionary boomed like
a cannon in the night:

{}``All kneel or be chastised, for now enters this hall his Holy
Piousness, Olorius Menamon the IVth, Deliverer of Aconia, Pacifier
of Zaelin, Proxy of the Gods, Procurer of the One True Law, and Castigator
of the Outer Territories. Kneel or be chastised!'', the last words
uttered with the gravity of a holy commandment, the obvious threat
to be carried out with ruthless deliberation if needs be.

At once and in concert, the whole of the crowd, including servants,
Protectors, Ministers, the whole of the Disciplanarium's staff, any
and all figures of authority, military, civil or religious, including
the Chief Functionary, kneeled on both legs and bowed their heads
deeply and solemnly, as if in wholehearted prayer.

The workings of some kind of large mechanism involving gears and other
mechanical contraptions rang through the audience hall, and the massive
copper tinted Gates of Leor opened slowly but steadily, revealing
the radiant form of the Castigator, breathtakingly dressed in the
formal robes of his Office, a deep crimson colour dyed in the blood
of heathens and heretics, solid golden runes written in Helica Pretoria
adorning the hem, and the Book of Law covering its surface with silk
threads so fine that the artisans went blind from the effort. And
above the robes, an immaculate platinum breastplate, with no carvings,
etchings or any decoration whatsoever. On one side, hung Urtis, the
Mace of Judgment, the Castigator's long ago chosen tool of enlightenment
and battle, that was said to have cracked as many heretic skulls as
there are stones in the walls of the Disciplinarium. Indeed, some
claim the very same skulls have been used in building the majestic
building, as a morbid reminder that All is Law.

The Castigator strode with a steady pace down the central lane where
a raised block of marble floor had appeared in sync with the opening
of the Gates. The only audible sound was the sound of the Castigator's
boots, a simple, utilitarian set of metal plated boots a soldier would
wear, finely polished, but otherwise uncommon. When he reached the
dais on which the Seat of Office stood, he surveyed the crowd momentarily,
sat down and clapped his bare hands once.

{}``Stand and confess!'', the Chief Functionary bellowed sharply,
and the crowd complied smartly and fervently:

{}``All is Law!''

And the Castigator echoed back the mantra in solemn ritual, his voice
carrying unusual depth and mesmerizing melody for a single man, however
powerful and unique he may be.

Those that saw the Castigator and had not been granded such an honour
before in their lives, were immediately left awestruck, and some of
them broke down weeping, pious fervor instantly occupying their hearts
and minds. Those that had been so blessed before, did not immediately
stand but rather silently prayed with tears welling in their eyes,
before being able to stand upright. The people that kept closer to
the Castigator, his immediate entourage, the Ruling Council, and his
guard intoned the holy mantra reverently and resumed their places
and functions.

The Castigator then addressed the crowd which stood there reverently,
their excitement and waiting evident in their glittering eyes and
tense faces:

{}``I shall call you my children, for I am like a father unto you.
I guide you, protect you, offer you learning and sustenance, like
a father for his child. I ask you: Does not the Ministry keep a daily
watch for the heretic, the heathen, the lawbreaker? Does it not preach
the Law daily, for the continued enlightenment of all? The Army, does
it not safeguard our lands, from enemies from within and from without?
The Procastinators, do they not wisely guide your every day lives,
always watching over you, less you stray to a horrible path with no
redemption in sight. I ask you again, am I not like a father unto
you all?''

The Castigator's voice turned from a sweet melody into a harsh pragmaticist's
staccato tone, then back in a wavering, almost pleading tone, evoking
sympathy and familiarity. The Chief Functionary struck down his distaff
on the granite floor once, and spoke aloud while nodding condescendingly,
bearded chin almost touching his chest:''Aye!''

The crowd followed in check, the audience hall reverberating from
the loud voices of what now seemed to be almost a thousand people.

{}``This then I tell you as a father, for the betterment of us all,
for the glory of the Pantheon, by the start of the new moon, in two
weeks time, the Holy and Righteous Armies of the Outer Territories
shall march into the Widelands, to bring enlightenment, cleanse the
land, and finally make the Land of the Gods whole, as is their mandate.''

His voice rang true and clear around the chamber, his message inviolate
and final, a decision that was to be carried out, not thought upon
or discussed, but a matter of fact that he had set in motion with
but a few of his words.

Most of the Ministers, Generals and other officials of the Disciplinarium,
were apparently surprised, though they instantly recovered a measure
of composure and if one had not been eyeing them constantly, they
would have looked easily unperturbed by the announcement and its implications.
The Procastinator Militant in his exquisite armor and fine silk sash
was at a loss for words, and opened his mouth wide-eyed as if in protestation
but his disciplined service and training so far kicked in, and barely
managed to save himself from embarassment by nodding in the last minute
and simply saying {}``His Piousness has spoken''.

The other members of the Ruling Council, the Archminister, the Patriarch,
and the Noble Representative had their gaze locked on the Procastinator
Militant, as if waiting for a sidestep or slip that would bring him
crashing down in a most shameful and undignified way, an affront to
the significance of the night. The Noble Representative, dressed in
a simple green robe of the practical sort, his chest adorned with
the signet brooch of House Remis, suppressed a grin at the nearly
unforgivable blunder that would have definetely incurred a public
lashing and a year's donation to the Ministry, let alone probably
kill the Procastinator Militant's carreer.

{}``I have indeed! Now feast, enjoy and praise the Gods!'', the
Castigator raised his arms in jubilance and smiled broadly, the atmosphere
in the hall warming up in the blink of an eye, people starting to
shuffle around, seeking food, drink, or whoever they had been talking
with before the Castigator entered the hall. The announcement of setting
out to pacify the last wild region, the Widelands, was definetely
going to spur debate, however hushed, controlled, and unchallenging.

Lord Umsepyre Remis, the Noble Representative, was seated in the council
in order to speak on behalf of the noble families of the Outer Territories.
His was a purely consultatory role, expressing current views among
the noble houses, informing the Ruling Council of the ebbs and flows
of power, wealth, and status, as well as the reactions and thoughts
of the nobles on affairs of state, religion and Law. 

Even though he had no voting rights in the Council, his input was
often quite impossible to receive otherwise, any network of informers
too crude in comparison. The Castigator seemed to consider him a quite
valuable asset, judging by the special dispensations recently attributed
to House Remis. In return, Umsepyre Remis was the best insightful
eye and ear the Noble houses had concerning the inner workings of
the Ministry and the Castigator, and solid knowledge of what went
on in the Disciplinarium sometimes could buy things money could not.

His was a unique position where he could not be accounted for practically
anything since he was not part of the decision making process, but
was amply able to exchange information and insight as he saw fit to
better suit his House's, and person's, continued survival.

He smiled brightly at the still uneasy Procastinator Militant, and
made a gesture to straighten his shoulder-high long black hair, in
an almost too bland demonstration of cool confidence.

The music that would accompany the dancing acts had started with a
brass fanfare, and soon settling in a soft string melody, oddly acented
here and there by flutes and bass drums, the dancers performing choreographed
scenes from the Mythos, reliving the handing down of the Law from
the Gods to men.

The Patriarch noted the interest on Lord Remis face who was more than
visibly enchanted by the dancers' performance, while the Procastinator
Militant had hurriedly called for his chief aide, more so in order
to look busy and industrious, rather than because actual operational
planning could be done at thiss time and place. The Archminister was
indeed busy on the other hand, a trio of scribes jotting down notes
and letters to be sent immediately, notifying key personnel of the
imminent rush of preparations that needed to start as soon as possible.

{}``Lord Remis. A patron of the Arts should be more circumspect in
his admirations, don't you think? Some may misinterpret your artistic
admiration for mere lust. And lust is a sin, Umsepyre.''

The Patriarch's tone was precipitously balanced between the whimsical
and playful and the vehemently dangerous and cunningly suggestive,
his mouth and face a rigid, expressionless mask. The man was an almost
complete mystery to Remis, he had to confess, his remarks and suggestions
always leaving a hint of sourness.

Remis was not incisively put off, and managed to respond appropriately,
though the effect on the Patriarch seemed to be minimal at best:

{}``The day we look upon the Mythos with lust, Your Reverence, is
the day all sin will be revealed.''

{}``Ah, quoting Law. I see. Am I making you uncomfortable, Noble
Representative? Is there a reason you would care to share?'', the
Patriarch was now smiling genially, an almost fatherly expression,
eyes darting around over Lord Remis face, genuine worry apparent on
the craggy old man's face.

{}``No, not uncomfortable Patriarch. I would say, curious.'', said
Remis, without looking directly at the Patriarch.

{}``Curious? Of the coming, final campaign?'', the Patriarch ventured,
raising his eyebrows and hinting he knew better than that.

{}``No. Of you, Your Reverence. You are always, shall I dare say,
fleeting.'', Ursempyre turned and look the Patriarch directly in
the eyes, seeing cold pinpricks of blue that reminded him of icy death
and men disappearing in a watery grave.

{}``Oh well then. Don't let me spoil your idea of me.'', the Patriarch
answered, head and gaze locked in front of him, and walked into the
crowd, his figure soon blending in to the point of disappearance.

Remis was left trying to ponder what went on in that man's head when
the Castigator's voice reawakened him into reality:

{}``Ah my dear Ursempyre, is the Patriarch causing you trouble?'',
the Castigator touched one of Remis' shoulders, a rare gesture of
unmatched camaraderie. He must really be the Castigator's favorite
to ascend. Ursempyre turned and bowed slightly, careful with his words
as his body language.

{}``Trouble is what a drunkard might cause, or a hapless wife. The
Patriarch feels like a man who causes death, Your Holiness. He seems
always to be, so detached.''

{}``Ah the toils of the Church. Holy Communion with the Gods can
be too much sometimes. However gifted and trained. You say he reeks
of death? I say if the Gods will it, I shall follow. Won't you, Remis?'',
the Castigator's look had been transfixed on him, awaiting a sure
and clear answer, all doubt dispelled. Ursempyre indulged him accordingly.

{}``All is Law, Lord. Unto death and beyond. For the glory of the
Pantheon.'', Remis recited the last phrase of the Oaths, and crossed
his hands over his chest, further reaffirming his loyalties.

{}``You are a good man Ursempyre. Now, drink! It is a night of feasting,
and joyous celebration. Won't you join me?'', the Castigator's eyes
suddenly turned ablaze with fear and wrath, menace under his voice.
Ursempyre was taken aback, only not visibly, and simply bowed and
said:

{}``Of course, Your Piousness.''

The Castigator broke into laughter, as if having heard a joke noone
else could, and started the rounds of the audience hall, like a perfectly
good host.

Ursempyre followed close behind, not so happy of his singular position
as he would have normally been. He took a silver cup filled with meade
and drank it in one go. It would be a long night. 


\section{The marble road}

A new moon filled the night sky. It was hunting time for owls, wild
dogs and wengals. He sat down on the cold, humid grass, legs crossed,
walking stick leaned on his left shoulder. He rummaged through his
ever lightening knapsack for something to eat. With little fuss, he
managed to come up with a meagre meal: worrain berries and a piece
of goat's cheese, still fresh and spongy.

The people in the village he went through last were the usual sort
in these parts: animal herders mostly and a few wheat farmers. Simple
people that prayed solemnly to the Gods, every day, wishing not for
riches, power, lands and wives or other things of vanity beyond their
grasp. They wished their child be spared of illness and harm, their
herds be still whole by nightfall, the snow be light this year, and
the rain plenty before harvest.

Fools then. He could feel sympathy like a man with sight feels for
a man in the dark who has never seen golden fields in springtime and
the piercing depthless eyes of a young maiden, but no more. Even blind
men wish for light to shine yet before their lives are ended. These
folk, just tread on, and never think otherwise. He wouldn't spare
them more than a passing thought.

His road was about to take him through the Widelands, an unforgiving
place. Few had made the passing, and even fewer were left sane and
unscathed in body, soul and mind once they did. There were stories
and legends about the Widelands, that were passed down from generation
to generation, in many languages of many folk. Most of them he determined
were the superstitious tales of simpletons and madmen, mere fantasies
for common people to spent the grinding, toiling days, thankful of
their safe, ordinary, almost pitiful lives. Some were invariably twisted
second and third-hand tales of those who ventured into the accursed
place and came back, surely not wholesome and well.

Fact and fiction were interwoven tightly in such accounts, but some
probable, shared truths could be distilled from the broth of rumblings,
mutterings, and assorted hearsay that permeated the inns and taverns
all across the Territories. There were even a few written accounts
by people who seemed to have genuinely made the journey, and live
to tell the tale and become rich and famous in the process, but most
of them seemed to be very talented liars and writers, the distinction
in their works of little significance. Only the Tale of Umberth could
be counted as less than nefarious, since he set out with a hundredscore
of men, of whom barely three survived: Umberth, his esquire Esphalon
and a woman, then only a child, who Umberth claimes was found wandering
alone, mute and dark-skinned, almost pitch-like.

Umberth spent the rest of his days and sizable fortune trying to organize
further trips into the Widelands, his few unsuccessful efforts presenting
the girl as living proof of his sayings of underground cities, height-defying
towers and huge arching constructs, but the girl could not even speak
her name. Her skin color was a singular phenomenon that more often
than not provoked the wrath of the Ministers and the aversion of crowds,
speaking of heresy against the Gods and the bastard offspring of heathens
and beasts of old that the earth unveiled from time to time to test
the faithful and attract the blasphemer.

Not long after the Ministry declared her as well as Umberth heretics,
an unrelenting manhunt together with the watchful, ever pious and
dutiful believers of the Pantheon bore fruit and Umberth and the girl
were beheaded and their bodies burned to ashes in Pyr, in the winter
of the third pacificum, 153rd annum, 51 days before the Solstice. 

Esphalon, his esquire, was sentenced to silence unto death and exiled,
never to utter a word in his life again, his tongue cut out. He then
wandered the northern lands, surviving on little more than worms and
lichen, possibly wishing death for himself. In those lands, he came
upon a small tribe of nomads, wild men that knew not how to herd sheep
and work metal or clay, with wooden spears and no command of fire.

From what Esphalon later describes, when he was trapped and caught
like animal prey, probably destined for a savage death, perhaps a
sacrifice or plain and simple cannibalism, the savages noticed while
they searched his clothing a small plaque of metal, hand-sized, with
no discerning characteristics other than a circular notch in its center,
a memento from the Widelands. When they noticed that, Esphalon was
greeted among them as an equal, a member of the tribe lost and found.

He soon realised the importance of that plaque, and was further mystified.
The savages used it to produce light and heat, by placing an ordinary
stone, even a pebble, inside the notch. Pebbles and stones were the
most widely used since they could be found in abundance in the northern
rocky steppes were these people seemed to roam unhindered. Sometimes
they used animal bones or pieces of wood, mostly whatever was handy
at the time.

Invariably, the said item slowly seemed to shrink and then vanish
into thin air in a matter of moments, and then pleasant sunlight and
radiant heat would emanate from the plaque for hours to come.

All this and much more is written in Esphalon's memoirs, a corrugation
of the things he saw and experienced in the wildlands coupled with
what little time he spent with the savages. The plaque remained with
the savage men who would not part with it in any way, and Esphalon
reached an isolated fishing harbor facing the Pangalor Sea, half-dead
from starvation, exhaustion, and lost a leg and three fingers from
frostbite.

It was there were he wrote the Pangalor Scrolls, the single half-consistent
work that contains enough about the Widelands to make it useful in
crossing it, and perhaps uncovering more about this mysterious land
and living long enough for others to know as well.

His reverie was broken by the cries of a nighthawk swooping in to
catch his prey, somewhere out in the star-lit fields of short grass
and poppies swinging gently in the night breeze. He marvelled at the
arrogant precision and expert flying of the nighthawk, which made
no effort to catch his prey unguarded, but instead seemed to announce
indifferently it was about to kill and then feast on a helpless, doomed,
animal soul.

He smiled thinly and then got up, fastened his knapsack, picked up
his walking stick and set off down the marble road. He looked overhead
and saw the nighthawk soaring triumphantly, a small snake writhing
in its death throes, captured in its beak. The nighthawk cried into
the night once more, and then disappeared over the distance, its faint
silhouette mingled with the starry blue and black sky.

The marble road was glistening as always, unmarred by war or nature's
wrath. He started humming an old tune his grandfather used to sing,
and soon he was lost behind the first hill.


\section{Under a livid sky}

When they had eaten, after Amonas conferred with Philo, the burly
man said he would be back soon. Hilderich had relaxed somewhat, but
he could not for the life of him fathom the two men. It might be just
as well. When he had seen whatever it was they wanted to show him,
perhaps they would stick true to their word and release him, let go
and consider him a friend. He wasn't sure how exactly a person who
planned overthrowing plots of heresy could afford to just tell people
then let them go and trust their good souls to tell not, but he had
a feeling it involved some kind of pain, extortion, or good old fashioned
fall-assisted life removal.

It just struck him odd. He felt he could trust this man, but his mind
reeled at the prospect. I must be going slightly mad, he thought to
himself. He was starting to think of ways he could snatch that keystone
and make a run for it, but the more he thought about it the more stupid
he felt. They had already outran him and outsmarted him once. Certainly
he was no match for either one of them in single even unarmed combat.
He now belatedly wished he had taken some time to practice his body
rather than spent most of it knee deep in curatoria.

What mattered was that for now, he decided his best chances lay in
playing along, and seeing what it was that they promised would change
his mind and heart forever.

The faint dusk had just given way to a pure, clean night. As he lay
there in his cot, he noticed Amonas had lit a large candle, and was
reading what seemed to be letters. He noticed the man had been rather
drawn into his reading material and decided against indiscretion on
his part. Amonas through the corner of his eye though was aware of
Hilderich mild scrutiny and without lifting his head from his reading,
he said to him in an inviting, conversational tone:

{}``From my wife. She's with child.''

{}``I didn't mean to intrude, I just noticed you were very much occupied.'',
said Hilderich almost apologetically.

{}``Still thinking of how and when to escape Hilderich? You are more
lively than you already know, friend. There will be no need.'', Amonas
looked him levelly, his gruffy voice adding an air of authenticity
Hilderich hadn't noticed so far. Perhaps he was being afflicted with
a disease of the mind, something in the food. But he felt alright
otherwise, he did not believe he was poisoned or otherwise tampered
with. He just could not believe he was having a normal conversation
with a person who was either mentally deranged or emphatically dangerous,
a heretic nonetheless!

Hilderich's thoughts were interrupted by Philo who entered the small
room, hood still on his head. He nodded to Amonas, who in turn said
to Hilderich:

{}``Come. You shall see for yourself soon.''

Hilderich stood up and Amonas offered him a hooded cloak, somewhat
more fine and more elegant than most. As soon as he was wearing it,
Amonas was instructing Philo to lead them on and out into the city.
Amonas would be trailing, and Hilderich would be in the middle, evidently
a small precaution on their part should he feel a sudden urge to run
like hell.

Philo lead them through somewhat shady alleys, light from far away
lamps barely reaching them. They angled left and right, as if evading
unseen stalkers. Perhaps there were some indeed. He would not be able
to see them or smell them until too late in any case. So he trudged
along. His knowledge of the city streets was perfunctory at best,
but even he could discern that they were walking towards the Disciplinarium's
hill. Even from a distance, the sounds of music and festivities could
be heard, and a light show of fireworks seemed to be underway as well.

Hilderich mused inspite himself:

{}``Are we invited then?''

Amonas smiled wryly and urged him forward, while Philo turned his
head around and stabbed Hilderich with his eyes. His attempt at humor
went unanswered.

The more they approached the Disciplinarium, the more care they took
in their approach, triple checking alleyways, hunching over shadows,
their feet as light as cats', not a sound other than shallow breathing.
When they reached the base of the Disciplinarium's hill, Philo signalled
them to stop dead in their shoes. He let a guard patrol vanish behind
the curve of the hill's base before urging them to rush for a certain
part of the slope, where the shadow of the aqueduct overhead would
conceal them.

They did so with a dancer's grace and reached the grassy part of slope
Philo had indicated. He took out a small knife from his belt, and
Hilderich watched in still surprise as he dug out a small piece of
tuft. A metal dial was to be found underneath, which then Amonas proceeded
to twist left and right accordingly to some whim or unknown turn.

Without a sound a small tubular opening appeared above the dial, large
enough for a man to fit inside but only in a prone position. It led
into a shiny metal pipe or tube of some sort with a downward inclination
whose other end was obscured in darkness, and judging from the light
stream of air wafting through, it was a long pipe indeed.

Philo nodded to Amonas and said,''I'll go on ahead.''

Amonas patted him on his left shoulder and grinned:

{}``Are you sure you'll fit?''

Philo was already sliding inside the mysterious and intriguing metalwork,
and muttered in a low voice what must have been a friendly obscenity,
if such a notion exists. Amonas then ushered Hilderich inside, imparting
a word of advice:

{}``Arms glued on your body. Count to thirty and then take a deep
breath and hold it. You're not afraid of water are you?'', Amonas'
low gruffy voice barely revealing of hint of worry.

{}``Water?Why?Count to thirty? Fast or slow?''

{}``Go on then!'', Amonas had to shove Hildrich inside, his protests
becoming dangerously loud.

{}``You really plan to kill me!''

Hilderich's terror was evident in the shrill quality of his voice.
It had all happened too quickly to try and hold back, so when Amonas
pushed him down the tubing, he tried to follow his advice, held his
arms stiff to his body and started counting to thirty.

The tubing angled downwards pretty soon and Hilderich felt he was
riding a children's slide, with the slight difference that the end
of the slide was nowhere in sight, and mad, possibly delusional, quite
certainly heretic thugs were shoving you down into one.

Hilderich heard a splash echo dimly in the metal tubing, and was suddenly
reminded of Amonas instruction. He wasn't sure if he had counted past
thirty or not, his terror and anxiety mixed with his confounded thoughts
a recipe against keeping calm and cool-headed. He filled his lungs
with air as long as there was time anyway, and just when he started
to think he had grievously mistimed his breath, he splashed into water.

The feeling was one of quite shock, the cold water encircling his
whole body, seeping through his clothes, assaulting his ears and nostrils,
as if trying to enter his body without his will. His eyes had closed
instinctively, then he opened them slowly, searching for the open
surface, his hands wobbling uncertainly, before a small primal fear
of running out of air urged him to swim upwards, towards what seemed
to be a faint source of light.

Within moments his head was clear of the water. He exhaled momentarily,
then breathed in gasps until he could find his normal breathing once
more. An outstretched hand seemed to be offering to help him out of
the water. As soon as he realised it belonged to Philo, he heard another
splash of water roughly behind him, and turned his head to look even
as Philo pulled him to somewhat dry land.

It was Amonas, who seemingly more accustomed to the area, needed no
help and within seconds was among them, tiny rivulets of water still
running down his leather vest, thick drops of water falling from his
forehead.

{}``Let us move. Hilderich, I would beg your silence. We are relatively
safe as long as we are silent.'', Amonas said in hushed tones while
holding Hilderich by the arm, in a friendly gesture.

{}``As if I've been screaming my lungs out. I'll keep my mouth shut.
Are we near whatever place you think will change my mind? Are we below
the Disciplinarium, by any chance?'', said Hilderich, while vainly
trying to squeeze off some of the water. 

{}``We are. But we still have some way to go.Come now, you will dry
yourself later.''

They had fallen inside a natural cistern in a small rocky cave. Strangely
enough, light seemed to seep through some cracks in the walls. With
a closer inspection, Hilderich saw the cracks were more akin to lichen,
faintly wet to the touch, but rough like sand as well. A sort of crystal
formation seemed to lie underneath such spots. Philo nudged him onward,
cutting his examination of the peculiarity short. It reminded him
of some lesser kind of curatoria, that his master was not particularly
interested in, and thus were only lightly studied.

But the light they gave off was indeed enough, even though only barely,
to walk the gently curving twists and turns of what seemed to be an
extensive network of caves, an almost ant-like structure, deep underneath
the Disciplinarium's hill. He noticed the steady, purposeful stride
of Amonas before him, and the dim blue and violet light that imparted
a grim hue on everything around and including them. A feeling of eerie
wariness started to seep in Hilderich. At length he tapped at Amonas
shoulder, who paused, turned and looked directly at Hilderich, an
expectant look on his face but not a sound coming out of his lips.
Hilderich asked in a low voice:

{}``Are we lost?''

Amonas did not answer, but rather resumed his walk, taking them through
caverns small and large and paths that could not be retraced unless
with a map. The further down they went, the warmer it felt, and soon
walking in the caves felt like a warm summer trip in the country.
Hilderich had by now lost track of their approximate depth, direction,
or distance of travel. Perhaps they were not lost, but he essentially
was and thought he would be unable find his own way back if the opportunity
presented itself. What good would it do though, since the tube was
meant to go down, and not up? Just as he let out a sigh of hopelessness,
he could discern light pouring out from the next corner, and feel
their slight descend leveling out. By now his clothes felt almost
dry, a slight feeling of dampness remaining in his feet and arms. 

The light grew more intense, a clear intense light like sunlight.
But sunlight, so deep under the earth? Not to mention, it was still
night outside. When they were only a few paces away from the corner
that shone with light, Amonas threw up one arm, palm open, indicating
a halt. He leaned over the corner of the rocky cave, and peered carefully
for almosta minute, making sure they were safe. Hilderich was not
in a position to know what he was keeping them safe from, or whether
or not the possible threat was imaginary or real. 

Amonas nodded that everything seemed as it should be, and walked out
into the light. He blinked his eyes flinching as they adjusted to
the brightness, and then gestured with his hand to Hilderich:

{}``Come. We are here.''

Hilderich stepped forward as well, the bright light forcing him to
instinctively cover his eyes with his right arm, the small peep of
its shadow the only shelter against such sudden illumination. A fragrant
waft of air rushed around them, as Philo joined them as well, and
all three of them were slowly walking towards the light. As Hilderich's
eyes finally adjusted to the light, he could see the rocky cave all
around them give way to a smooth white surface, much like porcelain
at first glance.

These strange walls seemed to extrude themselves from the strata of
cave rocks, as if totally alien and utterly old in origin. Then Hilderich's
eyes wandered a bit more and then his gaze rested on the center of
a huge chamber apparently made from the same white-ivory material.
The chamber was cylindrical in shape, and from top to bottom a wide
pillar which incredulously seemed to be made of pure light stood,
a sight that defied Hilderich's senses and logic.

{}``What is this place?'', Hilderich asked Amonas, awestruck, gaze
still locked on the pillar of light.

{}``This is the first pillar of truth, Hilderich.'', Amonas said
while edging closer to the pillar, Hilderich a couple of steps behind
him, his face brightly illuminated, but his eyes unflinching. Philo
stood at the rocky entrance to the chamber, with his back indifferently
turned to the wondrous sight in front of their eyes.

{}``The truth? What is this, Amonas? Is it Ancient?'', Hilderich's
tone had a far-away quality, as if he was mesmerized, his mind off
to some deep trench of thought.

{}``It is my friend. A working piece of technology of the Ancients,
buried deep under the Disciplinarium. What does that tell you, friend?'',
Amonas was looking intently at Hilderich's calm and entranced face,
his every word glistening with expectation.

{}``What it should. This must be preserved. Documented. Studied.
A Curatorium be built around it, scholars from around the lands should
visit and..''

Amonas grabbed Hilderich violently from both arms and attracted his
total attention, his voice free of constraints:

{}``Studies! Scholars! Can't you really see what this means?Think
of your Mythos!Think of your precious Law!''

Hilderich was visibly shaken as if coming around from a waking dream,
eyes rolling around trying to come to grips with his surroundings
and the man with the gruff voice in front of him, shouting at him
and looking ready to snap him in two if he said the wrong thing.

He asked him to think of the Mythos, and the Law. The Law was the
established religious canon that had replaced traditional secular
law thousands of years earlier. The Mythos was the recorded history
of the Gods handing down the Law unto the forsaken race of man, to
save it from destruction and withering, to help men reach their Gods
at the Time of Conjugation. 

The Law instructs man that those he knew as Ancients were the manifestation
of evil. That any tool or work of art or science whose workings cannot
be seen with the naked eye or touched by naked hand are containers
of evil. That there is only this land, and none other. That the lights
in the sky were put there by the Gods, to make nights more bearable.
That death is irrevocable and permanent, and those who do not uphold
the Law, will be shunt forever, their souls kept away from the Gods. 

That any man, woman or child upholding belief in the Ancients or their
works is a heretic, a vessel of corruption. That the Ministers, uphold
and teach the Law to the people. That the people, in turn, devote
themselves and their lives to abide the Law. That the Patriarch declares
additional Law as he sees fit, never in contradiction to the basic
tenements, and he alone chooses his successor. That a Castigator rules,
a man of wisdom and strength to lead the people, protect them, enlighten
the heathens, punish the lawbreakers and the heretics, and offer praise
and glory to the Gods.

That is what the Mythos and the Law say. Hilderich, though he did
not consider himself the religious type, had always been careful not
to attract the ire of the Ministers. He had memorised the Mythos and
the Law, as part of his training and education, as is the case with
most who are fortunate enough to be allowed to read and write, since
he was apprentice to a Curator, a guild of men with special dispensation
to hold and maintain approved artifacts from times past, some of which
have nefarious origins, keeping strict indices of what is stored where.
Only ministers and certain officials of the Disciplinarium or the
Army, and then again only after special permission is granted for
expressed purposes, can be allowed to even view a Curatorium from
the inside. To think that he was so honored. And to think that now,
he is looking at what seems to be a working example of heretical,
evil, technology.

{}``They know about this? The Disciplinarium?'', Hilderich asked
feebly.

{}``Of course they know Hilderich! They have always known! Do you
think this is the only place they have access to with such an artifact?''

{}``But, what does it do? And what proof do you have they know about
things such as these? Much less use one?'', Hilderich was argumentative,
but slightly unsure of his words, his voice wavering as if he was
shaking from the cold.

{}``Always the hard way, eh Hilderich?'', said Amonas and shook
his head in silent disappointment, nodding to Philo who winked back
an acknowledgement.

Hilderich was considering the magnificent simplicity and awesome sight
of the pillar of light in front of him, a bright pillar of sunlight
not as blinding as the suns, with a hazy rainbow of all colours on
its edges, a pulsating haze of tiny pin pricks rushing through some
sort of invisible shell. He was interrupted somewhat violently by
Amonas who neatly but decisively pushed him towards the pillar of
light.

{}``What are you doing?Is this safe?'', Hilderich seemed to protest
in principle only, his mouth voicing concern, but his feet offering
little resistance.

{}``Don't really know, it's only my second time.''

At that answer Hilderich's face took on an expression of exacerbated
disbelief, but only for a moment, because an instant later, both his
and Amonas' figures were vanishingly thin and elongated ghostly forms,
and then all that was left of them was a smooth scent of uwe stew.

Philo turned and looked at the pillar of light, gave a derisive rough
kind of snort, and continued his vigil, unperturbed.

\bigskip{}


The first thing that Hilderich felt was a sinking sensation, like
the inevitable pull and grasp of a whirlpool, the dreaded voids of
the sea that claim ships and men alike. Then he saw an explosion of
light, swirling walls of light running up to meet him face on, and
then nothing for what must have been less than a heartbeat. Then light
filled his senses totally, even his smell. He could have sworn, he
could smell light. Then his eyes somewhat adjusted to the radiant
sea of white enfolding him, and he could make out the outline of his
hands. The first thing he heard was the strange chirrups of birds
probably. The second was a gravely voice with serious undertones,
Amonas' voice.

{}``Take a step forward Hilderich. Don't be afraid now.''

He did and was left standing there. His eyes insisted that he was
outside, on what seemed like a clear summer sky. How he could have
in an instant walked past an underground cave as well as the grasp
of night, he was unable to answer, not without gibberish coming out
of his mouth. After a few more moments had passed, while playing back
what had just happened, Hilderich was finally able to ask:

{}``Where are we?''

Amonas smiled as he was pointing at the single yellow sun in the sky,
and simply said:

{}``I'd love to find out.''


\chapter{Fulcrum}


\section{A long and winding path}

The mountain grew ever more unkind. Its many bare faces looked down
upon a lonely figure, slowly but surely making its way through rock,
gravel, low grass and loose dirt. His hood was down, revealing a stern
but humble face. Care lines dotted his forehead, and one could easily
spot he was not a man prone to laughing easily. His face was adorned
by a beard grown out of necessity, not choice, and his thin long hair
was unkempt, a few wild strand jutting in strange directions.

The wind and the rain were thankfully absent on this day, awarding
him the leisure of trudging along the mountainous path with only his
sore feet and stiff legs to distract him from his effort. Indeed,
he paused once again to rest for a moment, let his blood flow freely
in his legs and feet, and take a moment to pray to God for his good
fortune.

He sat down on the naked rock, his buttocks well used by now to such
discomforts. He touched his forehead with one had and brought out
a small piece of knotted string with the other. His lips then moved
in a silent orison, asking for more of the same good fortune, and
perhaps a bush or nut-bearing tree from which to gather some much
needed food.

He had not seen or heard signs of goats or other mountain-dwelling
animals for days now. There was still some grass on these slopes,
so there should have been herds or families of animals feeding. Perhaps
there were many more richer plains and slopes far below, or in plateaus
his path had not taken him through. Perhaps it was pure chance that
he had not seen a living soul, neither a bird or lizard and certainly
not a goat. Perhaps it was his God, testing him for purity of heart
and strength of purpose, to steel him further in order to come through
the always perilous journey of Pilgrimage.

He was living on certain insects that were still to be found if someone
knew how and where to look, and a few roots he had been able to identify
as edible. The further deep inside the mountain range he trod, the
stranger and more different the life he met. At first the trees started
to become bulkier, more water rich, taller and greener. Then the animals,
he noticed, were more stout, fatter, their meat sweet and richer in
flavor, its color a vivid red, not like the dark, stringy meat of
the animals he was used to.

It was a sign, he decided, that with every step closer to God's Lands,
the land was graced by his favor, and the animals were fat and felt
no hunger, the trees and plants grew tall and proud, the birds soared
high and their voices were sweet as honey running down a child's mouth.
It was His work, all that was abundant, and all that was good.

This part of the mountains seemed to have fallen from His grace, whether
as another trial, or for reasons only He could entertain. No matter,
since his wisdom permeates the earth and the sky, he thought he could
only accept and never wonder, for that way lay madness, and the fall
from grace into pits of despair and malevolence.

He felt he was attuned, resonating with the earth below, the sky above
and the stone all around him. He reached into his small sack, and
with no effort produced a small circular pendant, a thin slice of
white marble or porcelain cast around a black mat surface, smoothe
and cold to the touch. He held it firmly with both hands for a while,
looking intently at its black surface and then a thin sliver of green
started to pulsate on the black surface, a green line starting to
form from one edge of the small black circlet, and ending on another,
to form a straight path, like the invisible brush of a painter kept
stroking the same line, always in the same direction.

He looked at the thin, green line of light, and then looked at the
faint mountain path that zig-zag through the ever rockier slopes.
His path was true, that much he knew. His pointing stone had not failed
him before, and neither would it fail him now, not on his Pilgrimage,
not while his faith was strong, and his prayer warm of heart and soul.
That he knew, and little else would come to matter.

He looked at the lands resting below him. The great northern plains
could be seen far away, a faint grey haze slightly discernible under
a thin sheet of fog. And then rolling hills of auburn slowly lifting
off the ground as if the very hand of God had touched and pinched
the lands, his handprint faintly echoed in the timid, graceful slopes.

Between the foot of the mountains and the hills, lay a deep gorge,
a wild river running through it, twisting and turning as far as the
eye could see further to the east, its flow coming from somewhere
deep inside the mountains, further up north, further than the lands
where his people roam, where the snow never melts and the suns always
hide behind the clouds, where neither man nor beast can endure for
long.

A fleeting sensation of wonder filled him, for the works of God were
magnificent to behold, and his Pilgrimage a unique journey of faith,
beauty, wonder and duty. The honour he was blessed with was indeed
so great he could have never thought it possible, much less aspire
to it. Nonetheless, he was on a Pilgrimage to the Land of God. He
was the Pilgrim, the one honoured to pay homage to the Land of God
and the final resting place of their forefathers. 

To him lay the duty of bringing back a Holy Forge, to pluck one out
of the very famed Garden of Wonders! His eyes were suddenly lit at
the very thought, even as his body still dully ached from the many
hardships his peregrination had knowingly brought upon him and would
bring him still. But he ignored all that which occluded his mind and
he imagined himself, standing amidst the Garden of Wonders and quenching
his thirst from the Unending Spring.

It would all be more than worthy of the pain, the cold and the rain.
Just to lay his eyes on the Veiled Gates, he would willingly give
his life. But he cannot, and will not, until his Pilgrimage is complete,
and his people have their Holy Forge renewed. Oh, the joyous wonders
he is yet to behold, not just the earth and the rivers and the mountains
all wrought in unquestioned wisdom, but the craft of God Himself,
right in front of his eyes, at the touch of his hand, from silver,
and stone, and sand that never crumbles, or faints, and never shall.

His senses brought him back to the cold reality. He still had some
good light left, and he should not waste it. Every day without a Holy
Forge was a harsh day for his people. His journey was still many days
and nights away from an end. Tarrying here in the middle of the mountains,
daydreaming like a young selfish brat was not at all what any would
expect of him, the one so honoured. His shoulders suddenly felt a
bit heavier with so much resting on them: the future of his people,
the life of the land, the children yet unborn.

A gust of the mountains cold and wholesome air seemed to have infected
him with renewed vigour. In seconds, he was already steadily climbing
up the steep, winding path that would take him between the two dominant
mountain peaks, and afterwards probably on a shallow descend to the
Land of God.

Those who had gone before him had followed the same path, and had
passed on word of their travel and their journey. What mountains and
ridges to pass, which rivers and springs to drink from, what strange
growths and roots to eat, where to feel safe and sleep unhindered,
as well as where to keep one eye open and your knife in hand. More
than a few had perished and left traces of their demise for the next
ones to follow, and more than a few times the people had almost been
extinguished, their last footprint upon the earth carried away by
the winds that know no sympathy or heed no pledge and care not for
the lives of simple men.

But God provides, and always will. As long as we have faith, as long
as we live our lives like we were meant to, taught to from father
to child, as long as we go on the Pilgrimage when time and God mandate.

These thoughts occupied his head as he toiled onwards, even though
under his thick pelted boots his calloused feet could feel every last
jut of rock and bit of gravel. This was him now, this described him
wholly. He was Pilgrim, meant to walk the earth until he reached the
Holy Place, and then whether or not he would go back to his people
and his previous life, that meant nothing. When he performed the Rites,
the Holy Forge would be with his people. And then he could walk back,
and live the rest of his days having witnessed the glory of God, teach
those that would come after him, and if it fate would have it so,
help the next Pilgrim prepare for his own difficult journey.

Perhaps, he thought, I am getting too far ahead in my thinking. My
journey is still far from over, and yet here I am, feet dead as wood,
legs heavy like rock, once again on the climb, all I've loved and
known left behind perhaps forever, and I am let myself be fooled by
visions of a future yet to unravel, me at its centre.

Selfishness. Ego. A sign of malignancy, a precursor to evil thoughts
and desires, accursed manifestations of Them. May God watch over his
people and lend him strength and clarity of mind and purpose. To think
such thoughts, when God was already pointing to the true path, when
everything so far had proceeded along according to his divine plan,
when the auguries had said it was a good time for a Pilgrimage. That
he was a good man, that he would be a true Pilgrim, one tat God would
accept.

He felt he had to cleanse himself with birch and water, pay obeysance
to his God with an offer of personal sacrifice. But he was already
on his Pilgrimage, what could he do now that would not interfere with
his holy purpose? He had no inkling yet, but he felt blood rushing
through his veins, feeling guilty, shameful, almost soiled.

He pushed harder, the slope turning into an almost sheer wall of rock.
The small, narrow path had degenerated into a granite crevace with
pockmarks and surfaces of chipped rock that one had to climb with
hands as well as feet.

In his mind, it mattered little, because he felt like he would grow
wings if he had to, if the earth was without warning removed from
his feet. He felt like he would grow gills and scales, and swim the
oceans of the world if the skies suddenly opened and poured all the
water of the world and the earth was covered in it.

He steadily put one hand after the next, hoisting his lithe and supple
careworn body slowly but surely, every step of the way a small death
for Them and their venomous influence that seeps into the hearts and
minds of the weak-minded and unfaithful, spreading over the poeple
of the earth like a rotting disease.

Perspiration glistened on his forehead, the small of his back was
damp as well, everu muscle and joint burning from effort and protesting
at every leap and move. But he kept going, his mind focused, his soul
shielded and armed, a searing force of pure light stabbing through
a heart of darkness, a pestilence of lies, deceit and wrong.

He was fighting Them, even as he climbed, and sweat, and toiled. His
whole Pilgrimage was a Holy War, he now knew, and this very climb
a fight. Like the War between God and Them, at a time before man.
The same war, a million fights, a million more, until God prevails,
until the faithful have had their share of blood, toil, and fighting.
Until then, he would climb for his people, His Faithful. Until then,
he would endure the forces arrayed against him, be they nature, men,
or Them, in one disguise or many.

He would endure and he would prevail. Not to become a revered one
among his folk, not to serve some delusional idea of a grandiose self
in a small world and an even smaller land, an even fewer people. It
was true, their numbers were dwindling, their women bore less children,
and their liver were becoming shorter. He would endure the hardships
of his path and the machinations of the enemies of God, for the good
of his people and the will of his God.

He reached out with one hand blindly, his face wearing an expression
of determination, a resolute, stout mask under which nerves flickered
furiously with jabbing explosions of pain and anguish, though for
no one else to know apart himself and God.

He grasped for a handhold in the rock, a fissure, a jutting piece
of granite or lime, but all he could grasp was thin, cold rushing
air. With another leap, his face was caught in the stream of air,
his hood wildly fluttering about his neck, sweaty locks of hair caressing
his face. He had reached the neck between the two peaks, and he could
now see a wide stretch of plain-like ground extending before him,
grim patches of grass, rocks and dusty gravel for days worth of travel
ahead of him.

He could feel the touch of God as he took the final step onto the
plateau, his soul drifted away by a divine wind, his aching body forgetful
of his aches and trappings. He felt light as a feather, in body and
soul. He remembered then the words of his Guide, his people's master
of lore and faith, their holy man:''And once you step onto the wide,
grey mesa, a gust of wind will greet you and lift your soul. It will
be God whispering in your ear, it will be a sign from God, that your
path is true.''

Indeed then his path was true, one of many perils left behind. He
let out a laugh, inspite of him, a laugh he would look down on with
contempt as blasphemous, but it was a laugh that welled from the soul,
a liberating act, a cry of thanks to his God, his protector, his ever
watchful Father. He started walking with a steady slow pace once more,
with what little light of day remained guiding him to a cluster of
rocks where he might find shelter for the coming night. Once he lay
down, he would pray to God and offer him his gratitude for saving
him from disgrace and keeping him on his true path.

And then he would sleep and dream of goat's cheese, berries, honey
and meade..


\section{Gossamer Twilight}

A faint aroma of cinnamon and rosebuds permeated the bedroom, a thin,
comely sweetness lingering in the air, inviting nothing but warm memories
and cherished moments to those it happened to touch. The first light
of dawn had just broke, the sounds of early morning in Pyr resounding
as ever. 

The Ministry Tower rang once, then twice, for the people to wake up
timely, and offer their daily prayer to the Gods. The scurrying feet
of water sellers and milk men could be heard from the balconies, running
through town, their carts filled dangerously with bottles and oversized
clay or wooden containers, selling their wares to those in the city
who could afford to. 

Celia woke up as the beam of sunlight characteristically bounced off
the gleaming, copper-skinned Ministry tower, as was usually the case.
Her long golden brown hair, tangled as it was from last night's fretful
sleep, resembled a flaming bush when the copper-tinted light cast
off the ministry's tower shone upon her. Her visage was one of a fiery,
avenging god-maiden of fury and destruction, an avalanche of wrath
rushing down upon the wrongdoers and evil-makers that dared incurr
her retribution.

But that was only a fleeting impression, for when she touched her
grown belly and felt her unborn child still soundly and safely asleep,
her smile was like heavenly orchards grew and bore ripe honey-sweet
fruit in the blink of an eye, all the goodness of creation coming
together in a still moment of time, a mother's smile, a power beyond
reckoning and imagination, all that in the creases of a beautiful
face and two comely lips. And she who had seemed as a terrible force
had been wondrously transfigured, into a mother bathed in sunlight,
radiating warmth and love, any hint of terrible awe a mere phantom
now in the eye of the beholder.

After a few moments of silent contemplation, as if communicating with
the foetus growing inside her, and some moments of simple indulgence
in smelling the morning fragrances and hearing the first sounds of
day, Celia threw her sheets away playfully and got out of bed to follow
her usual routine: she took her morning bath, and then offered her
own prayer to the Gods, thanking them for sending a man like Amonas
to her, thanking them for conceiving her child, carrying it this far,
and praying for her husband's safe return, and her child's first cry
into this world.

She was then startingly taken by the fragrant smell that had gently
occupied her senses ever since she opened her eyes, and felt almost
strange for not immediately taking notice of such a beautiful scent.
With a familiar way, she tried to trace the source of the fragrant
smell that seemed to seep through the walls and pour from balconies
and windows. She peered over her stone balcony, more and more people
starting to wade through the streets, the day starting off in its
usual rhythm. The bakery on the corner of their street had only begun
to unravel steaming loafs of bread from its oven, no sign of sweets
and caramels and other sugar treats. It was only to be expected, on
a Watchday. Who would dare cross arms with the Law? Much less for
a little taste of sugar. And Cerpiem, the stall vendor, who sold almost
anything from thin sheets of writing paper for the rich and affluent,
to small pieces of gum for the children. He had nothing of the sorts
laid out.

Who was the mischievous rascal then? And even more so, ignoring the
proscriptions of the Ministers? Her curiosity and her happy, playful
mood took over. She would set off to find this little rebellious soul,
and why not, share some of the forbidden fruit if he would share it. 

She put on her simple flowing green dress, the silver hairband Amonas
had fashioned for her once she had known she was carrying his child,
and went down her stairs and onto the busy street.

She knew Amonas would be somewhere here in the city of Pyr, but it
was much safer for both of them if she knew not exactly where he was
and when he would be coming back. It was for the best, that was what
he had told her, and she felt suddenly saddened by his absence, staying
away from each other while he could be so close was stressful and
felt plain wrong. Like a child instructed not to play for fear of
falling down and hurting.

She was suddenly brought back to her immediate reality in a breathtakingly
surprising way: a hefty slice of cinnamon breadpie was standing right
in fron of her, and old lady Rovenia was holding the outlandishly
little plate, a knowing smile written across her care-worn face, bright
green eyes that had seen more than a lifetime's worth of sadness and
happiness twinkling with a child's mischievous glee.

{}``You! Shame on you lady Rovenia! On a Watchday too?'', said Celia
as she took the offered plate with one hand and ushered lady Rovenia
through the footstep of her house's door, which she had barely walked
out from herself only moments earlier. Lady Rovenia said nothing,
but kept on smiling, and Celia thought she could perhaps hear a little
snort or giggle as the old lady came inside the house, with a breadpan
full of hot sweetness and fresh aroma carried under her arms.

{}``I knew you had been spenting the last few days all alone, poor
woman, husband away working for hard earned coin. He is working away,
is he not? Dear Gods, I hope it's not some fiendish tale to fool you
and damn your family in brothels and gambling houses?''

Celia broke down in laughter which was not easily contained and as
an added trouble, a morcel of breadpie still being chewed was inadvertently
stuck between her stomach and her mouth, prompting her to start coughing
fervently, but at the same time, folding herself across her belly
as much as her situation permitted, laughing, choking, and coughing
at the same time, if that was at all possible.

Lady Rovenia was dumbstruck at Celia's reaction but she was quick
to act as well. She found a small alcove on Celia's kitchen wall to
set the still warm breadpan, and then hurried to the drinking bucket,
and poured a large cup of water for Celia to wash the troubling piece
of food down.

Celia, still trying to recover from her attack of hysterical laughter,
and still coughing, took the proferred cup of water, and drank in
large gulps. Some of the water ran down her cheeks and neck, some
of it she spluttered while coughing, but most of it washed the breadpie
down, and she could safely breathe again, now laughing because of
the way she nearly choked. Lady Rovenia seemed terrified, a pale look
of worry drawn all over her small wrinkled face, but managed to say
in a quavering voice full of concern:

{}``For the Grace of the Gods, girl! I only brought the pie to sweeten
your life, not end it like a candle blown away!'', and lifted the
cup away from Celia's smiling lips.

{}``Lady Rovenia, had I known you had such a roaring imagination
I would have kept you company more often than not! Amonas, in brothels,
and gambling? Perhaps you'd suggest he was a drunkard too?'', Celia
said smiling widely, pulling up a chair for the old lady to sit in
first, and then one for herself, back straightened out and carefully
managing the space between her belly and the kitchen table.

{}``You thought it funny and a product of wild imagination but in
my many years I have seen the like of what you should never believe
possible, probable or in any way imaginable, but I have, and there
is one thing men can be trusted for: trust them not!'', the old lady
said while waving one finger wildly as if casting cantrips with an
invisible yet powerful wand of magic like the one in fairy tales and
ancient stories. 

{}``I have nothing much to say to that, my good lady Rovenia, only
that Amonas is not just any man, and had he been such a one, I would
not have loved him for once, much less marry him or carry his child.'',
her voice sparkling with admiration and notes as if made of honeydew
clinching her every word.

{}``If that's what how you feel alright, but men change. Be wary
girl, that's all I say. How are you coming along?'', the old lady
smiled heartily and touched Celia's belly lightly, delicately, with
hands that had offered the same kind touch perhaps hundreds of times,
over the years.

{}``I can have no complaints. The child is mindful of me, I can sense
it. It sleeps mostly when I do, and his kicks and restlessness are
but tiny nudges. Some mornings I feel sick, and no amount of clean
air helps. But I gather that is only natural.''

{}``It is girl, but tell me? Do you eat for two? Do you feed yourself
properly or do you think an appetite is woeful feeling? Do you sleep
well enough?'', the old woman's voice sounding loud in a preaching
tone, almost scolding the young mother to be, her one hand slapping
the table surface mildly, in time with her little less than accusing
questions.

{}``I eat well enough, dear Rovenia. I eat as much as I feel full
and then some more. And then I can hardly breathe, and have to pace
around the house or go outside for a little walk for my stomach to
settle down properly. Have no worries about me or the child.'', said
Celia, clasping the other woman's hands in her own.

{}``I'll worry if I like to and you can say none otherwise. How are
you sleeping now? You seem like you had a troubled sleep, your hair
is tousled and your skin has an off tone. You're not sick are you?''

{}``No Gods forbid, no. I was twisting a bit in my sleep, and then
when I woke up I smelled that breadpie of yours and came rushing outside
to find out who.. Oh dear me, I'm such an awful host! Should I start
a fire, boil some water for uwe or keplis right away?''

{}``No need to rush dear. I'll help you along. I might be old but
I'm not an invalid, not yet Gods forbid.''

{}``And we'll munch some of that nice breadpie of yours. Tell me
then, aren't you scared of the Watchday's proscription? What if someone
with malice in his heart goes and gives your name to the Ministers?
And with such a strong smell, perhaps they'll smell it of their own
accord, down to the Tower itself!''

{}``My lovely girl, if I was scared of such things, I would have
died of fear a long time ago. As it is, I couldn't care less what
they think of me and whether or not there are rats above the sewers
as well. And what would the Ministry do to a poor old lady like me,
for baking a sweet bread on a Watchday? Lash me like the heretic I
am? No, child. Their mind works like one of those steamers, all brawn,
power and rashness. Let them think the world can work like one of
their machines, all steam and air. Arrogant and self-important. Dear
me, I can feel my blood rising in my ears.''

{}``Now then no need for your blood to gorge like that, you are safe
here in our house, I promise you, I swear in my unborn child, what
I cherish most. Calm yourself, and have some of your lovely breadpie
to sweeten your sour taste.Come.'', Celia said to Rovenia in her
most calm and assuring tone, the one she felt really calmed Amonas
and brought him peace and serenity.

{}``I'm fine girl, I'm fine. Let's forget about them, since they
can't appreciate a fine breadpie. Now, eat!'', said the old woman
with a disarming smile and they happily ate away.

And with that they passed the time until the evening arrived, exchanging
stories and tales, but mostly Rovenia telling her the story of her
life in many small parts, whatever little story she felt most appropriate
at any time, whether it be funny, sad, or in some curious circumstances,
both.

Celia was drawn to the old woman's tales like a kitten to warm milk.
At first she sipped them slowly, and then drank them all in as if
catering for an insatiable desire to listen to everyday tales of an
old midwife and caring neighbour. Celia thought old Rovenia would
be quite fitting for the task, and the old woman lovingly knocked
her door from time to time to check how she was doing, if her health
was anything less than adamant, and if the baby kicked and felt lively
enough.

From time to time she scolded her for not taking good care of herself,
not eating enough quality meat, and there never seemed to be enough
greens in her kitchen for that woman's taste. She insisted on Celia
eating lots of eggs and milk if she could cope with it on a daily
basis, and drink lots of uwe, said to be good for everything, from
the blood to the bones. And when the child would stirr troubled, the
best medicine would be to sing to it, sing from her heart, whatever
soothed and pleased her.

When the old lady finally left the house, it was more from polite
awareness of the time that had melted away from the hearty discussion,
almost like one between an anxious soon-to-be grandmother and her
pregnant daughter. Celia had never known her mother, and her father
was killed in the Pacification of Zaelin, little enough to understand
almost nothing around her, but old enough to remember her father kissing
her goodbye for the last time, his kiss at once somber and warm to
the touch. She had grown up with her grandparents, who loved her more
than possible and raised her better than a daughter. They had died
before they could see her wed.

Celia offered the old lady to stay for the rest of the day, provide
her valuable insight and experience in cooking the next day's meal,
but Rovenia said it was unhealthy for the baby to stay inside for
long, and that she should definetely go on a walk while there was
still any sunlight left, since nighttime in Pyr was less than a proper
place for an expecting woman, the light of the lamps leaving more
to the shadow than seemed wise and proper. So, like most ladies at
her age, she would retire in the warmth of her small house, and pass
the time weaving jedoons and other pieces of cloth that might come
in handy, or not. And then she laughed briefly but genuinely, and
went next door, leaving Celia to tend to her own household in peace.

She did just that for some time, idling away at brooming and cleaning
the house as much as her straining back allowed, and then she brought
firewood from their small cellar, and paused for a while before starting
a fire anew for tomorrow's meal, perhaps some stew, perhaps a broth
of beans and greens, a recipe Rovenia had suggested while they were
idly gnawing at her breadpie.

She stood with a few small pieces of firewood cupped into her arms,
when the thought flashed in her head unwanted: she was out of uwe,
again! She left the tiny logs aside on the small kitchen table and
went upstairs to fetch her small purse of coins. She then made sure
she hadn't actually started off an untended fire, drew her broadly-hemmed
cloak around her and darted off towards the market, light waning,
the falling dusk painting the distant encircling hills in an orange
and purple hue, some of the rooftops in Pyr shedding a brown-yellow
sheen.

The city crowd on the streets was shifting towards its nocturnal aspect,
the ones that rarely venture outside if the suns still abound, and
rarely crawl back to their domiciles before dawn is about to break.
Loud song and cheers, sounds of merrymaking and laughter could be
heard at least once in every street that she passed on her way to
Ves, the farmer she knew was her kin, a cousin in fact, and tried
not to fool her like others in the market did.

She wasn't sure if she would make it in time, for at this hour Ves
as well as almost everyone else with farms or animals to tend, had
to leave for his farmstead, eat and rest, before getting up in the
middle of the night, watering whatever plants needed so, then harvesting
those ready for the market, and then loading up his cart and off he
would be to the market once more to make some coin, for his wife and
children not to beg like some who were cast adrift in the unfathomable
torrents of fortune.

She could see them, dishevelled beings, sometimes indistinguishable
from animals, sometimes bringing a sore tear to her eyes. Some she
helped as she could, a loaf of bread or her daily bottle of milk.
Sometimes she would leave a plate of fodd for those that drifted throughout
the city and did not just await their end at some dark corner of the
market, either stamped upon, trodden by mistake or not, sometimes
hunted as a passtime by men drunk enough or too clearheaded to care
as they ended their horrid lives, bringing on a fate only fit for
nightmares.

As she turned the last corner before she reached Ves' usual stand,
she could see she was too late, Ves as well as everyone else having
departed for the night, leftover fruit and vegetable stalks amassing
on the cobbled streets, a sour acrid odor wafting all around her.
If nothing else at all, she had indeed taken a walk, though at an
inappropriate hour, and she should be getting home before long.

As she turned around to start walking back towards her house, she
was frozen where she lay, when a mailed hand seemed to stretch from
utterly nowhere, some shadowy crevice, some chasm in a wall she hadn't
noticed in her dimly lit surroundings, a surreptitious figure that
seemed well-disposed towards her, or else she would be already lying
in a pool of her blood, for what little coin she carried in her tiny
purse, or things worse than an untimely death that she dared not imagine
while carrying her child still in her womb.

{}``Lady Celia, be still and fear not. I am Kin, and I bear news
for you: Philo has been arrested, but your husband has not.'', the
man's voice steady, straightforward, serious and business-like.

{}``And Amonas?What of him?Is he dead?'', she managed to utter in
total disbelief, a well-practiced phrase in her head, her moment of
fear given body through her quivering low voice, a stutter barely
avoided, her lips trembling, her eyes narrowed down to small ovals,
what little blue was left exposed to the light of the lamps, flashing
with terror. Her hands had instinctively gone to her belly, hugging
it closer than ever, as if she feared the child would be needlessly
drawn away from her, a life unborn for a life given.

{}``Hush mylady, we know not. But no body to be found, or a trace
of cloth, we can be sure. Have hope, Lady Celia. And let not a soul
know of this.'', the man left a hint of consolation in his voice
but none of that would be enough for Celia now.

He was gone as silently and instantly as he had appeared, through
the unseen folds of the night, a messenger in the dark, grim and hollow
thoughts in his wake.

She ran back to the house, tears running down her pale cheeks. Blood
had left her face, and coldness crept in like endless tides of water
running under a door. The laughing crowds became a sorrowful noise
in her mind, a weeping in her heart. She ran up to their bedroom,
feeling the baby stir uneasily, as if it knew something was amiss. 

She lay down on their bed, put on her wedding gossamer tirval, and
wept until she could weep no more and her tears dried and her numb
mind sent her into a merciful sleep.


\section{Of the Sun and Moon}

{}``Well that's what you said last time as well. Is it that difficult
to accept the fact we are now, thanks to your efforts, terminally
lost?''

Hilderich was picking ineffectually at the withering bark of a large
oak-like tree, swarms of ants running up and down its length, tiny
flecks of dead wood on their backs. The tree was turning into ashes,
returning into the dirt, one very small piece at a time. Hilderich
was quite fascinated by what he was seeing all around him and that
was probably why he had not broken down in hysterical cries because
of their mishap. Which was also why he could blame Amonas for their
predicament, in a steady, calm, matter-of-factly and somewhat detached,
distant tone of voice. Half his mind was infuriated, close to bursting
actually, because Amonas seemed to have had inadvertently stranded
them on \textsl{somewhere. }The other half of his mind was trying
to make connections between the flora and fauna of this general area,
place, whatever one might call it, and the various curatoria he believed
he remembered having some kind of relation in part or in whole.

Indirectly, this was Hilderich's way of coping with the problem in
hand, partly to offset his mind and unburden it from stress, anxiety,
and generally what he had learned to consider counter-productive emotions.
And on top of that, he was actually trying to help in his own lateral
way, by trying to identify anything he might be able to, based on
whatever curatoria he had studied or seen in his unfortunately short
and, recent events not withstanding, uneventful apprenticeship. It
seemed that apart from superficial resemblances and some generic common
traits, he had arrived at no particularly useful conclusion. For the
time being, he reminded himself silently and thoughtfully.

Amonas was sitting at a partly exposed root of a gigantic kind of
a tree he had trouble accepting that was real, even though he had
been sitting right there, on the same spot, for the better part of
an hour, silent, thoughtful and certainly perplexed, even though Hilderich
had spared little of the last hour discussing with Amonas, or simply
looking at the man's face, which would be enough for the even the
most socially inept, slow-witted and sentimentally detached human
to understand the man was deeply troubled, almost morose, and not
without good cause.

Insects abounded in this humid environment, the likes of which he
had not known existed, not even in the southern-most bogs and marshes.
Sweat poured from their bodies incessantly, making their every movement
a sticky, messy business. It was the heat. The heat of desert combined
with the water, the moisture, of a lake or river. They had seen no
river whatsoever from the top of the hill where they had emerged,
and had come upon no body of water in their blind search so far. The
humidity and heat of the place was overwhelming; Amonas thought they
should devote most of their time and effort into simply staying alive
for the time being. That meant finding a source of clean, fresh water,
preferably by nightfall.

Amonas' mind buzzed incessantly with the same thought; it was indeed
his fault. Hilderich had been right. His initial purpose when pushing
Hilderich into that infernal pillar of light, that machinery truly
in its makers' image, was to force him to see the truth. The first
time he himself had stepped through that beacon of light, he had been
instantly transferred into a huge, deep cavern, walls of solid metal
jutting out of the bedrock, an incomprehensible labyrinth of large
metal pipes, interconnections, spines and all manners of weird machinery
and constructs the likes of which he could not believe were made by
mere men, but rather by Gods, or their offspring or servants. He had
seen words in High Helican he had not seen before, dangling in the
air around him as if stamped with thick light on a giant spider's
gossamer web. He had seen visions of gruesome death, savagery and
bloody toil, endlessly replayed as if it all was a theatrical stage,
and he was a lucky viewer.

He had seen so much more he needed to forget as well, but could not,
in fact, dared not forget, lest the hatred for their jailers, captors,
these madmen, would diminish, ever so imperceptibly as to make one
think that it was still raging as blistering as ever. The words he
lacked to describe the sickening mob of rulers who moved freely about
like a sickle does unto stalks of wheat. No such euphemism of words
like dictator, or killer, would really suffice to describe them completely.
He felt like he would have to leave that to someone else, since in
the end, he might not even be able to slit their throats in person.

Without knowing, he seemed to have been gripping his knife from its
blade, so intensely he had cut himself, a small rivulet of blood and
sweat running down his wrist, droplets of rosy red falling down onto
the constantly wet ground. Ever thirsty and never quenched, be it
blood, water or both, these new lands seemed to feed on desperation
and sweat. His focus returned to the immediate reality around them
and felt the sagging weight of the situation. He had to have faith
in himself: he was a man of action, and he had already decided to
forfeit his life if it came to that. Others were capable of carrying
out the same mission as he was supposed to. If he could not do so
in the end, he only felt it was wrong for Celia, and their unborn
child. Lovely Celia.. 

He looked up to the alien looking sky, so familiar but so different
at the same time. He could remember his days of ignorance and blissful
youth, riding in the countryside, galloping fast and hard as if the
world's end was rushing right behind him. And the sky had this strange
quality, a light blur, a haze of wonder, looking as if it was a mere
ceiling he could reach up and touch as long as he wished it hard enough.

He would not perish here, he decided. It was as simple as that. He
would keep his promises and find his way back, to make things right.
To free his fellow men. To live life anew, reborn. As fresh and innocent
as his firstborn would be. It was time to act, secure any means of
survival in this strange land, acclimatise themselves quickly, for
who knew how long their journey back would take, and have faith, in
themselves, and their purpose. It should be Hilderich's now as well,
even though he cannot yet grasp the extend of the lies, deceit, and
exploitation. He should be able to put Hilderich to some good use
as well; the man who was a little older than a mere boy had good qualities.
He was smart and perceptive, suspicious but not predisposed, simple
but not simple-minded. He would be fine. They would be fine. After
they worked out some of the issues involved, though.

The suns were wrong, for starters. There was only one sun here, and
it seemed brighter than usual, but smaller, the hue of its light an
almost lime green. And then there were the towers, or spirals, he
wasn't sure how to describe them. Hideously tall and thin towers in
pairs, a low crescent shape adjoining them. They kind of reminded
him of bull horns, if he had to describe them more plainly. Far away
in the distance, were the haze allowed it, they could make out not
one of these monumentally proportioned constructs, but a dozen or
more, in regular intervals. They were fascinating, but largely irrelevant
at this point. He had thought about broaching these matters to Hilderich,
but he decided against that for now. The detailed intricacies of their
whereabouts, their actual location, the climate and topography of
the region, were merely academic issues if they would not provide
a small shred of actual information that would lead them to somehow
going back to Pyr, or any recognisable place for that matter, in the
Territories or elsewhere in the world.

He was still absorbed in thought, eyes piercing the tall canopy of
thick foliage, the huge volume of the surrounding trees standing like
rocky pillars between two worlds: their own earthen cradle which defined
them by preventing their return, and the other, the heavenly shell
of a world that they were yet to reach, the world back home, their
haven.

Hilderich literally slapped him back into the real world, the palm
of his hand wet and sticky, his cheek flush from the hit.

{}``Are you listening? Are you here? Gods help me, he was insane
to begin with now he is catatonic! Amonas!''

Hilderich was shouting now, still thinking Amonas was day-dreaming
or far worse, had finally lost his, so he thought, fragile mind. As
he swung his hand back once more to deliver another slap, Amonas turned
his head ever so lightly and looked him straight in the eye, and simply
said:

{}``Don't. No need. I must thank you, actually. I was thinking. I
was, overly engrossed in thought I must admit. Were you calling me
out for long?''

Hilderich was genuinely surprised at such behavior. Had Amonas been
catatonic, he would never really respond and then he would be left
alone, and with his survival skills and his latest round of luck,
perish in this steamy cauldron. If he was indeed mad, he probably
go berserk and snap his neck like a twig if he was lucky enough. It
seemed now that he was neither. He was simply, as strange as it seemed
to Hilderich at the time, hard at thought. He sat down in front of
Amonas, on the leaf strewn ground, wet and muddy, a continuous hint
of rotting vegetation waxing and waning from the faint whisps of air,
a fitting reminder of what happens to idle life. Hilderich cleared
his throat and while still looking at the ground, toying around with
a small branch, idly looking at the ground, he asked with some reluctance:

{}``Amonas. This is real, right? This is not a trick, not some very
elaborate way of forcing me to join whatever it is you meant to in
the first place? Is it?''

Amonas bit his lip and answered, palms outstretched, an ornate ring
of silver and copper catching the eyes of Hilderich for the first
time. He seemed to draw some breath, then pausing briefly as if he
intended to say otherwise before nodding in acceptance and telling
Hilderich:

{}``It's real Hilderich. That is what has dragged me down in thought.
I am sorry Hilderich, my intentions were quite different, and certainly
did not involve getting utterly lost, especially at such a moment
in time. I know I have failed you so far Hilderich, so I will promise
you nothing. I can only offer you my help in order to find a way home.
Preferably, while we still draw breath.'', a bitter smile forming
on his lips, his head turning to look once more at the thick foliage,
hoping to catch a glimpse of the strangely immaculate, perfectly cloudless
sky.

When Amonas looked down again, he noticed Hilderich had fallen on
his back giggling almost maniacally, his knees bent haplessly in a
comical angle, arms folded across his chest, hands clapping with the
whole of his palms. Amonas frowned quizzically, Hilderich's bizarre
reaction to his statement leaving him unable to understand or much
less respond at all.

{}``\textsl{You?} Offer to help \textsl{me?''}

Hilderich sat upright, legs sprawled in an uncomfortable position,
his hand pointing at his own face in sheer disbelief, his voice a
falcetto. His face looked suddenly harsh, unforgiving, out of place
with his normal self, and then he raised an accusing finger at Amonas,
saying to him in a calm and studied manner, as if lecturing a man
of lesser intelligence:

{}``Not to insinuate that you have done a very poor job so far, but
please, indulge me. How can you, help me, find a way home. Are you
perhaps a magician of the old tales? I think not dear friend, since
then you would be in posession of a cone-shaped hat, talking gibberish
even while you sleep, though I cannot yet vouch for the verity or
not of that particular supposition. Another possible way in which
you might be of help would be that you are, in fact, a fallen angel
of the Gods, who has yet to use his superlative powers in my favor
because he is as ever trying to teach me a lesson in humility and
religious awe, lest my soul is eternally condemned in Catharteria,
Damnation, the Twelve Wheels of Fire and so on. You seem to be missing
your wings, shield, divine aura, and angelic face, so I would say
no, you can't help me like that. I briefly considered that you might
actually own this particular piece of land and are indeed \textsl{dying}
to offload it to a dimwhitted fool like myself who might mistake the
extreme humidity, unbearable heat and overflowing vegetation for marshlands
suitable for cotton, or something equally senile. Without trying to
hurt your feelings or vested interests in a manner most ungracious,
I regret to inform you I find your selling points lacking and will
not be following up on your offer. Now, unless I am mistaken in all
of the above, and unless you have some button or pillar of light that
does the opposite of what brought us here, I dare say we are properly
doomed, and good as dead and finished. Other more vulgar expressions
pertaining to our present unfortunate situation come to mind, but
I will not bring myself down to such inestimable depths of bad taste
and linguistic ineptitude to use them like a debased wretch of lesser
stature. I will now honestly state my predilections, one, that I wish
to hang myself at the nearest opportune moment in order to escape
further unneeded physical torment under these circumstances, and two,
that I wish for my remains to be burned, as is customary under Law.''

All that Amonas could do, was blink, wide-eyed and at a loss for words.
Hilderich was resting his hands on his knees, cross-legged on the
wet ground, an air of finality around him, as if a holy avatar had
announced the end of the world.

Amonas shattered the uneasy silence with a question, uttered in complete
fascination, a glazed look of mock awe on Amonas face, his gruff voice
adding tremendously to the intended comcal effect:

{}``Are you sure you weren't studying to become a Minister?''

And with that, they both broke down in hearty laughter, the strain
of their situation and the accumulated fatigue almost vanishing as
if washed clear away. An invigorating smile graced Hilderich's mouth
before he answered in kind:

{}``Actually I had been thinking about it, but though I can handle
the dramatics, I am not too keen on handing people over to the procastinators
for spilling oil or eating sugar on a Watchday.''

Amonas thought there was a lot about Hilderich to muse over when time
and circumstance would allow it, but there were other more pressing
matters to attend to first. 

{}``Hilderich, I'm sure we'll have quite some time to exchange more
tales. But we have to attend to our survival first. We will need fresh,
drinking water. And something solid to eat, surely. Something that
will not easily spoil in this heat and moisture, this unlikely combination
of marsh and forest. But our priority should be water. At the rate
we sweat, we will surely suffer the most from its lack. And too soon
for comfort, I would wager. Speaking of which, have you noticed? The
shadows, they are strange.''

Hilderich nodded, then looked carefully around them, at the barks
of trees and small rocks and hanging green overgrowths. He looked
at what one would call a glenn if it weren't for the awfully wrong
conditions and the green overarching roof made from ostensibly ancient
tree branches. The canopy was a mosaic of green and brown hues, the
greenish light of the sun adding an emerald glare to the columns of
light that shot underneath, where Hilderich could see, the shadows
stood still.

Indeed, he noticed that the shadows had moved little or not at all
since earlier. He couldn't be sure, but he knew it was at least strange,
and probably another indication that they were very, very, far away
from home. He pointed at a broken log with his right hand, a tall
outstretched branch casting its shadow on a peculiar half-grey, half-bleached
stone.

{}``What time of day would you say it is, Amonas?'', asked Hilderich
while still pointing his hand in that particular direction.

After little deliberation Amonas answered casually:

{}``I would say about noon. But I could be mistaken. A few hours
ago, before we reached this place, night was well under way. And then
when we came here, it seemed like a bright summer day, the sun still
rising proudly. My body and mind long for rest, my sense of time should
be in disarray. But if I woke up right this instant, I would've thought
I overslept into noon.''

{}``That shadow was there when we went down from that hill and sat
here first. I remember because I imagined pouncing your head on that
rock.'', Hilderich added matter-of-factly.

Amonas frowned, but did not press the issue. Instead, he nodded in
silent agreement, then said with a careful choice of words, as if
musing on a worldly matter (which was perhaps on this particular case,
not an overstatement):

{}``Then.. If shadows stand still.. Does time as well? Is this a
limbo of sorts? A jail.. for our souls? If we return, will it be as
if no time has passed?''

Amonas seemed troubled by these newly found thoughts. Hilderich on
the other hand had no qualms in throwing Amonas interpretation of
facts out the window.

{}``That's nonsense! Even ministers would find that assumption idiotic!
At best! Master Olom would have you scrubbing the horses for a week
for even pondering such a connection! I mean, scrubbing! Flayed brush
and murky water for a week! Grooming a horse is no occupation for
an aspiring Curator, mind you! And especially the horses' parts where..''

Amonas had the decency to interlope and cut Hilderich in mid-sentence,
offering his timely excuse:

{}``I trust you will be more forgiving than dear Olom was and should
the opportunity arise, I will be more than happy to be accordingly
reproached for making such extravagant extrapolations. So what do
you think?'', his voice finely and expertly tuned to defuse Hilderich's
probable ranting and almost concede in a sincere fashion that he was
out of depth here and it would be more wise to let someone who knows
better find out what is going on.

{}``Tha shadows during the day are cast because of the suns' light.
So, when the suns move across the sky, so do the shadows follow in
hand and move accordingly. Would the suns stay still, so would the
shadows. It is not entirely without logic to postulate that since
we have witnessed only one sun, this is perhaps the reason for its
inability to move, and hence the standing shadows and the continuous
moon. There it is, a much more simple explanation which, as my late
master would have said, is usually the right one. The sharper, the
better.''

{}``Like a razor then, Olom's razor?'', Amonas grinned to show he
remembered the old man fondly as well, a shared memory they had yet
to explore.

{}``You could call it that, I guess. It might prove to apply in more
subjects of interested.''

{}``I hope it does. To me, simplicity is a virtue.''

{}``Indeed.''

{}``Now we know it will be noon for an inordinate amount of time,
is it not wise to assume that it will never be nightfall?''

That had not immediately dawned on Hilderich, and the revelation left
him looking worried and puzzled, more so because he had not followed
out his own train of thought completely.

{}``If that comes to be, then the heat will not dissipate, and this
will go on until we are able to return, the halflight, half-shadow
under this monstrous canopy, sweat and grimy mass of rotting leaves
stuck on our bodies. Or then again we might never leave this place.
The prospect of spending days or weeks in such conditions, whether
or not we will be able to go back, makes me want to once more consider
adopting an inherently expeditious approach to making oneself go away.''

Hilderich was seemingly more humorous than before, and his words were
not to be taken for granted, but Amonas had to admit to the fact that
this strange sun and taxing climate would make their efforts even
more strained and difficult that he had calculated. And still he feared,
they had no solid idea of how to get back home. Survival would have
to take precedence. And that meant finding water, not sooner or later,
but immediately.

{}``Come, we will find water.'', Amonas said decisively, and picked
up his pace towards a seemingly random direction.

{}``Under different circumstances that would involve my absence,
I would be impressedd by your optimism, but I would have to point
out that water is abundantly present, the problem being that it seems
to lie on either our own sorry selves, or the ground and the slowly
rotting plantlife it supports. How do you plan to go about doing that,
pray tell?''

Hilderich was already on his feet, following Amonas from close behind,
careful with his steps, avoiding what seemed the most grisly pathways
and wet spots that held soft matter of dubious origins underneath.

{}``We'll start searching where the plants look thicker, greener
and more lively. There should be some source of running water, at
least underground, like the places were we would look to dig a well
back home. We'll take it slowly, the more we exert ourselves, the
worse it will be in the end if lady Luck keeps running out on us.'',
Amonas said while plowing on ahead, working his knife in one hand,
hacking away any lush growths that proved to be obstacles in his path.

{}``I'm not very excited at what you are suggesting, but I cannot
think of anything better right now, so I'll just trudge along.'',
Hilderich admitted with a small hint of grudge in his tone, and an
almost imperceptibly condescending sort of nod.

A few hours passed, Amonas grinding their way through ever thicker
vegetation, now stripped naked to the waist, the heat and humidity
insufferable to bear with his leather vest and chain mail underneath
that. Hilderich wondered at how the man had suffered to carry all
that weight at all, never mind wearing all that in such conditions,
and only choosing not to when after they had been walking for the
better part of an hour. It seemed as if the man had grown literally
attached to his set of armor, or that its prolonged use had left indelible
stains on his body. None of those reasons, it seemed, stood to reason.
Amonas had simply not taken them off because he hadn't felt inclined
to. That probably spoke volumes for the man's tolerance threshold,
and what he was capable of going through if pressed, but Hilderich
thought he had no desire to learn, since he believed this whole experience
would be if nothing else, extremely educational and vividly remembered
if there was any afterwards to be had.

Amonas seemed to indefatigable. He had trod on through thickset lush
overgrowths, greens and all sorts of wild vegetation using his indispensable
knife, and had complained neither for the steaming heat or the breath-clogging
moisture. Hilderich had refrained from asking questions about the
reasoning behind their apparently random course through this probably
impossible to map land, lacking great physical characteristics easily
identifiable, used as points of reference. Except those twin towers,
or bull horns, or giant forks or whatever one might wish to call them.
The name would be indeed useless once one laid eyes upon them. Such
majestical structures, in the middle of this chaotic spread of plants
and everything else that should better be left unmet. They were roughly
headed towards the general direction of one of those structures. Structures
that mere men could not have wrought, but would have had if they could.

Hilderich's concentration was broken by Amonas triumphant voice, a
hundred or so feet ahead of him, still clearly heard over the distance:

{}``Water, Hilderich! I told you we will find water! Come! It might
not be cold, but it's not lukewarm either! Come!''

Hilderich felt Amonas was not unreasonably excited about his finding,
but he could not readily share the joy. His feet though he did not
complain were hurting, and his feet were a soggy affair, not the least
bit dry. His light boots had let all the moisture in, and their path
had guided him through many mud-soaked footings. He could feel his
skin was not up to the task, and he looked miserable to the bone.
Still, finding water was the first good thing that had happened ever
since that fine uwe stew, and that previous day seemed like another
age alltogether. He managed a defiant smile which he hoped Amonas
would not misunderstand, and carried himself to the small trickling
water source where Amonas was washing his face.

{}``Finally then. My mouth feels like a Ministry's rug, like everyone's
trod on it!'', Hilderich said jokingly, cupping his hands under the
small trickle of water running down through an old tree's bark, like
someone had fashioned it specifically for that purpose. Luck it would
seem, hadn't run dry just yet.

After they had managed to wash away some of the sweat and grit, and
more importantly, quench their thirst and fill their belly with more
than water than it could handle, Amonas took Hilderich by the arm
and suggested to him that they should try and make some sort of camp
here, near the water. The rain would be coming, he said, even with
the sun above, since there can be no water with no rain. And they
would have to keep dry, since that was how many folk in the sea died,
their bodies found almost dried out, dessicated, husks rather than
flesh. It was because water was attracted to water, and the water
in the body, the blood, the piss, the spit, all that water was drawn
away, to the river, or sea, or whatever larger body of water you happened
to be in. The rich got richer, even in nature, perhaps even in this
weird land as well. In any case, Amonas had convinced Hilderich that
it was wiser to stay dry, and it would be indeed a welcome change
in any case. They would have to build a fire to do that, and with
all that humidity everywhere, he could not for the life of him figure
out how, but they had found water, so they would build a fire too.
Or so Amonas said. He was very convincing, Hilderich reassured himself
before feeling unmistakably hungry, his stomach sounding like a cauldron
on fire with nothing inside the broth but water.

{}``We need to eat too.'', Hilderich admitted frankly to Amonas,
who nodded knowingly.

{}``I haven't seen a breathing thing yet, only biting insects and
that's no good if they drink your blood and you try to get back at
them. Besides, it wouldn't be worth it. No real river or stream to
try and find fish. We'll have to rely on you then, Hilderich. Try
and find some kind of root, stem or plant in general you think might
be safe to eat. I don't mean taste good, Hilderich. I really mean,
eating it won't kill us, not right off anyway. I know I can leave
you to it while I gather what wood and fiber I can manage to start
working on that small tent of ours. Have faith in yourself Hilderich,
we'll get back.'', said Amonas in his usual gruff voice, a friendly
tone that suggested and inclined more rather than ordered and pushed
around. It was the voice of a leader, Hilderich realised, a man fit
for the task at hand: keeping people basically alive.

Hilderich nodded, accepting the task though as with most tasks, not
knowing if he was really up to it. He turned though once before starting
to rummage through the thickset leaves and lush bushes all around,
and asked Amonas:

{}``I noticed we were roughly headed towards the structures we saw
from that hill. Do you have something in mind?''

{}``You noticed, eh?I thought you would. I don't have something particular,
just more of a feeling, an urge if you like. And to be honest, what
more is there to look around here? If there's some kind of a device
similar to the one that brought us here, we're better off looking
at one of those things, before scouring the whole damned land hoping
to blindly stumble upon one.''

{}``That's true.''

Amonas went about making a make-shift tent, and Hilderich finding
something edible. Within less than half an hour, Amonas had laid down
a few logs, half rotten and half dead, but good enough for the job
in hand. He had stacked them so as to make a simple crude roof, and
then covered that simple skeleton with snapped off fresh branches
and twigs, and overlaid huge thick leaves from the innumerable plants
available. He hoped these would suffice, and once he had massed enough
pieces of wet yet not soggy wood, he piled it down neatly in a firestack,
wishing his flint and stone would be enough to get the fire going.
He called out to Hilderich, to check if he had found anything. While
he received no reply for a few moments, the moment he started to feel
worried about his whereabouts, Hilderich popped out of a cluster of
bushes, with what seemed an armful of fine large mushrooms.

{}``There's more!Fantastic really!Of all places, renia mushrooms
this size, here!My grandfather would have a fit!'', Hilderich shouted
enthusiastically, walking over the stacked wood, looking for a good
place to leave his priceless armful of mushrooms, but displeased with
all available options decided to just stand there, a load of mushrooms
twice his size carried on his arms.

Amonas smiled, greatly pleased and mildly surprised, both for their
luck as well as the gleam in Hilderich's eyes, a genuine expression
of happiness, however transient and irrelevant in the long run, it
was good for morale. Hilderich's and his as well.

{}``Put those down and help me get the fire going.'', Amonas nodded
over the stacked wood.

{}``Oh no. I've never started a fire in the woods before. I'd be
useless. Always used a bottle of .. Oh, you might be inadvertently
correct in your proposal. I'm telling you though, you will be eating
the ones that touch this sorry excuse for a ground.''

Hilderich indeed lowered his body, almost in a squat position, to
put down the load of mushrooms as intact as possible, a somewhat neat
pile that did not immediately crumble when he let go of his arms.
He then searched through the numerous pockets in his vest, and procured
a small metal flask, which he proferred to Amonas with a radiant,
beaming smile.

{}``Gin. Fine grain, citrus taste. A distil of mine, from time to
time. Well, frankly, more likely when master Olom would be away on
important business. But, I insist, my distil. He never touched the
stuff.'', Hilderich's voice playfully mischievous.

{}``Well, you are more than meets the eye Hilderich D'Augnacy.'',
Amonas grinned while taking the small flask and dabbing with it some
more or less dry cloth from his own garments, then placing it where
the fire should be lit.

{}``Which reminds me, Amonas, you seem to know my name, though you
haven't yet very well met me. You told me only just yesterday, that
Amonas was the name I needed to know, and the rest would be revealed
to me in due time. I believe, the time is long due, wouldn't you say?''

{}``You are right, friend. I do owe you that and still more.''

Amonas was busy with his knife and nicely shaped and sized rock, that
seemed to spark properly. With a couple of more efforts on his behalf,
sparks flew into the gin-wet piece of cloth and the fire leapt out
as if beckoned by a spirit of old, rushing, blazing like it should.

{}``Amonas Ptolemy, friend to those that wish it, enemy to many.
Husband to one.''

Hilderich could sense the sorrowful note in his voice, his wife understandably
a part of him, part of his name.

{}``I'm sure she'll be quite happy to greet you on our return, will
she not?'', his tone uplifting, playful, a smile forming on his face
even as he put a nice whole piece of mushroom through a stick, getting
ready to roast it over yet undone coals.

{}``I am sure she will, just not as sure that she will be as happy
as I will.'', Amonas answered in kind, he too, skewering a mushroom
head cut in slices with his knifes through a stick of handy size.

{}``Well I'm quite happy around these lovelies here.'', said Hilderich
amusingly, gesturing at the small pile of mushrooms, with the hint
of an innuendo that would make master Olom instantly bash his head
with whatever in hand at the time of uttering.

They ate until they were full, and their sense of taste and smell
satisfied beyond mere hunger. The fire was burning well now, clothes
hung overhead with a well balanced piece of wood and some of the hanging
green ropes of vegetation seemed to dry sufficiently. Hilderich had
laid down on top of his cloak, feet outstretched, drying out close
to the warmth of the fire. Amonas had lit a pipe with what he thought
passed for uwe around here, and puffed away, lost in thought. He had
offered some to Hilderich who politely refused, and instead downed
a few sips of his own distil. As if it had been bothering him for
days on end, Amonas turned and asked Hilderich:

{}``Won't you sleep now? You must be exhausted from all this. I was
a soldier once, I've known similar hardship. But you? You should be
half-way home in a dream by now.''

{}``I know what you mean. I feel like a metal press was weighing
me down and now that I ate and laid myself to rest, it has been lifted.
I should have fallen soundly asleep, as you say.''

{}``And what's keeping you, Hilderich?''

{}``The moon. There's no bloody moon to fall asleep by.''


\section{Machina Segnis}

The suns seemed to have risen earlier on that day, or at least the
Castigator's people definetely had. At every level of office and hierarchy,
the living mechanisms of the Ministry, the Army and the Procastinators
were of singular mind and purpose. The Castigator had announced that
in two weeks time the wrathful military might of the Outer Territories
would be ready to march for war. Something that had not happened in
the past 25 years.

In every single office and chamber of the various organisations of
people and ruling institutions, the situation was almost the same,
if one would take into account the multitude of minor variations in
disciplinary strictness, interpersonal roles and affiliations, as
well as structural differences and the specific nomenclature of each
branch of service. 

Lesser officials busied themselves with arranging communiques, writing
down orders and manifests, then calling for couriers or perhaps taking
it upon themselves to forward the appropriate documents and even materiel
to their intended recipients. It would be anathema to any one in that
overwhelmingly complex machine of sorts that he should singularly
fail in the most simple of orders. If anything were to happen to this
whole enterprise, this majestic war footing, this Holy Campaign, issued
by the Gods, commanded by the Castigator himself, then it would not
be because a lowly clerk or young lieutenant forgot to sent out some
materiel recquisition form, or a call-to-arms teller. And indeed,
if this Campaign was to fail because of a human error, in such a catastrophic
way, before it even began to put itself into motion, everyone performed
beyond their absolute best to ensure that it would not be on their
account, on their watch.

Diligence was considered a virtuous characteristic, and most Law-abiding
parents tried to hammer that into their children if they had to, so
as to become proper people, upholding the Law whenever they could,
serving from any place in society they might reach up to. Those that
the Gods seemed to favor most, were selected to enter public service,
either as Ministers, Procastinators, or Army men, according to how
well they performed at the Agogeia. Select officials from these three
embodiments of rule and order taught at the Agogeia, schools for those
that could afford to become something useful in their lives. Once
the basics such as obedience to Law, reciting scripture, and fairly
simple counting and swordfighting skills were taught, the best of
each class of children were selected according to the inclination,
receptiveness, and skill they showed at the various tests and games
carried out for the very same purpose of separating the wheat from
the chaff, those capable of serving the people and the Gods. 

Some, those better skilled in memory, oration, the use of language
and emotion, capable of swaying their fellow students to their own
purpose, were further trained to become Ministers, further trained
into the teachings and trappings of Law, how to best interpret Law
according to need, how to teach, enlighten, and chastise laymen, and
how to impress and guide hundreds if not thousands of people as Law,
Ministry, and Ruling Council dictated. They would be responsible for
the enlightenment of the people, teaching them the Law, and helping
them avoid the temptations that would lead to blasphemy, heresy, casting
out, eternal damnation, and a most probably gruesome, demeaning execution
that would serve as a reminder and a lesson that All is Law, and noone
and nothing is above it. The Ministers would also tend to the daily
running of the Territories, as administrative officials, collecting
offerings, making amendments to the lesser decrees of the Law to better
handle the multitude of people and the realities of land of the living
required, with its economy, trade, and needs. The spending of coin
for public works would be decided and then dispersed accordingly to
those noble houses that could field enough manpower to make it happen,
like roads, bridges, canals, buildings, walls, mills and workshops
and every other resource that would the Territories grow and prosper,
for the glory of the Pantheon. And one of their number would be chosen
from the Castigator, with the blessings of the Patriarch, to be the
next Archminister, the one blessed to be the voice and heart of the
Ministry. 

Those of the Agogeia students who were energetic, athletic students,
showing exceptional stamina and strength, exemplifying martial prowess
with the blade and their bare hands, those who were blessed with possessing
a sharp decisive mind, had proven to be of faith unfailing and a stone
hard dedication to the Pantheon, those were chosen as fit for service
in the Army. Rigorous training in all the known aspects of warfare
was their only occupation until death, whether or not they were called
upon to act, kill or be killed in service to the Gods, they would
train with sword and spear, shield and horse, until old age came,
when they would carry on with training others of their kind, in matters
such as the planning and design of warfare where a mind should be
much more fit than the body. Their training started with single combat
techniques, with many different weapons, under different situations
and varying levels of duress. Then they progressed into squad tactics,
in the open field, against other types of units, like cavalry, or
steamers and artillery. And then they would rotate into the rest of
the units, for their training to be complete and be able to use everything
from their empty hands to a complex steamer machine, and be knowledgeable
in the weakness and strength of each one, being able to select the
best course of action and what kind of men, machine or animal it would
require to be successful. These were the core lessons they were taught:
strength in knowledge, success in adaptability, glory in death. As
they progressed through the standings of the Army, always according
to their merit and degree of success in their duties, always keeping
in mind their faithful devotion, through their accrued experience
they learned more about handling men, materiel, and equipment, organisation,
designing and planning with tens of thousands in mind, as one day
they might be called upon to lead the whole Army as Generals, in the
name of the Pantheon first, and the Castigator second.

Those that did not excel in anything, but showed average skill at
wielding a sword, and could learn enough of the Law as needed orally,
those did not learn to read and write like the Minister's did, nor
train further in order to excel into combat. These children were strictly
chosen for their ability to follow the letter of the Law, blindly,
unerringly, keep a watchful eye for signs of heresy and insubordination,
any element that was an affront the Pantheon, and the Law, anything
that defied the Law or its upholders in spirit or in letter. These
were the ever watchful eye of the Castigator, the arm that made the
Law reach into every heart, body, and mind, the Procastinators. Their
training was simple, crude, and effective, hammering into them the
utmost loyalty unto the Law, as well as teaching them how to use people
in order to learn all that there is, all that is going on, the rumours,
the happenings, the weddings and deaths, births and oath-takings.
Everything that went on not just in Pyr, but everywhere in the Territories,
they had to know. And if the need arose, they disciplined, re-enlightened,
or fetch to the Ministers those they deemed suspicious or genuinely
guilty of sin. And then they enforced the Law and the divine will
of the Ministry, unflinching, following the credo that All is Law.
And if they performed impeccably, surely they would have the honour
of becoming Procastinator Militant, part of the Ruling Council, the
left hand of the Castigator himself.

It was a structure that had been handed down from the Gods themselves,
so its purpose and form were Holy, and any talk of reform, change,
or deviation from the established was treated at best as blasphemy,
but usually as expressing heresy, and was treated accordingly by public
torture and death. None were exempt from such punishment, especially
the men in the Ruling Council, who were the paragon of Law itself
to all the people. Such a hideous concept was not unknown, that a
man in the Ruling Council would denounce the Gods by commiting or
speaking heresy, for it happened long before, in a past almost rightly
forgotten and excised from the Annals of the Territories, but still
lingering in peoples' memory as Shan's Betrayal, a myth to freighten
the children into obeying, a fable to instruct and put the fear of
the Gods into the soul of men.

Shan had been a General of the Army, at a time when the Territories
had not grown past the lands around Pyr. When the Ruling Council decided
it was time to enlighten the nearby shores of Urfall, Shan was reluctant,
at first. The story says he was publicly chastised, with a hundred
lashes to his body edging him close to death. Because he was deemed
an exceptional strategist and a peerless tactician, he was once more
asked to lead the armies that would enlighten Urfall, instead of being
stripped of office and rank and live on the streets as a beggar, given
the opportunity to redeem himself in the eyes of the Pantheon, the
people and the Castigator. He acceded, and the armies marched off,
gleaming in their metal armor, the blessings of the Castigator sung
over the Southern Gates of Pyr. Within a few weeks, Shan's armies
seemed lost, no message of the war reaching Pyr, and no messengers
from Pyr ever returning.

One day, Shan appeared over the hills encircling Pyr, and had with
him not only the armies he had marched off with weeks before, but
horses, and men and catapults and hellish contraptions that spurted
fire and death, from Urfall. He had spread the heresy to the armies
like the mythic whores of old spread disease, like the cancer that
spreads from the roots of a tree to its leaves and brings about its
death. And he had the Urfalli with him, their machines working in
unfathomable ways, the products of heretical pacts with the forces
of evil.

He reached the Gates of Pyr and demanded the surrender first and foremost
of the Ruling Council, and had the ineffable audacity to accuse the
Council of lies and crimes to the people, twisting the word of the
Law and speweing horrific untruths, in an attempt to poison the minds
of everyone in the city as well, promising that none of those who
surrendered willingly would be hurt in any way, and a fair trial would
be arranged for all, except the Council who would be executed after
their supposed lies had been exposed and their non-existent crimes
against men proved unquestionably. Such heretic lies had never been
uttered or thought of before, and never would again, their venomous
treachery so base, the Castigator himself is said to have cried in
desperation, for he had never thought a dearly loved brother like
Shan would fall from grace like that.

The city was utterly defenseless, save but a few procastinators and
old army tutors, and lowly farmers, herders, artisans and traders
that had not yielded shield, spear or sword not once in their lives.
The armies of Shan had cast away all form of decency, form or honor,
and turning into a heretic rabble, had begun to scour the lands, pillaging,
raping, and burning, before what they thought would be the grand feast
of Pyr itself.

But they never managed to sink their putrid claws and teeth into the
immaculate flesh of the City of Pyr, for it was protected by the Gods,
as are all their faithful and humble servants. For when the time was
nigh, and all seemed lost, the Castigator Hanul Ofodor the 1st, retired
from the halls of the Disciplinarium, deep into the Sacred Vaults,
where he and the Patriarch offered their blood to commune with the
Gods, and ask for deliverance in that time of need. For a day and
a night, while the heretic hordes of Shan looted and pillaged, and
while the outer walls of Pyr were about to fall, an angel sent from
the Gods appeared in their image, casting brilliant rays of cleansing
light, and annihilating the armies of Shan who had no other recourse
but to flee like the vermin they were. None escaped the angel's wrath,
who spread the cleansing fire to every last part of Shan's army.

When the Day of Redemption had passed, the City of Urfall and its
majestic harbor and proud workshops were all extinguished in a ball
of light so pure in its wrath that those who saw it with bare eyes
went blind, and would forever be praised in their lives as Martyrs
of the Wrath of the Gods, spreading the tale of Shan and what they
saw to everywhere they went. And such was the way the story was told,
from one generation to the next, as a reminder, even though the official
Annals never admitted or recorded it, for as much as it mattered,
it should never have happened in the eyes of the Gods, and so it never
had.

And such was the tale of Shan Lagus, the Betrayer, proscribed from
history, but alive in the memory of Law-fearing people, people like
the Archminister LaVasse, a wide, big-boned man, dressed in an opulent
surplice, holy texts in High Helican weaved around the sleeves, a
Seal of Office hanging round his neck in a pendant made of platinum
and emeralds, fittingly pure and clean to represent the qualities
of the Archminister.

He was presently at the Strategium Proper, in the company of the General
of the Army, and the Procastinator Militant, whose embarassing near-blunder
at last night's festivities at the announcement of the Last Holy Campaign
as it was officially now named, had not gone unnoticed, and had become
the subject of sarcastic comments and irony even at lower echelons
of Ministry and Army, but had only naturally been ignored by the Procastinators
as a whole.

The General of the Army had not been present at the event due to having
received news of the Castigator's decision from beforehand, and had
indeed spent the night hard at work putting his most trusted and capable
people together, rousing them up from their sleep in order to lay
down the priorities of planning and start orchestrating the massive
preparations involved in such an endeavour as a Holy Campaign. He
had of course learned of the Procastinator's Militant blunder, and
even though second hand accounts rarely manage to do justice, he had
exchanged knowing looks and smiles with the Archminister that had
gone largely unnoticed by the Procastinator Militant, a somewhat alarming
fact if one would care to extrapolate the level of the Procastinators'
vigilance from the qualities apparent in its most senior member.

The three of them had been there from before dawn, the Archminister
and Procastinator Militant arriving together though having rode in
separate coaches, having left from the Disciplinarium once proper
etiquette was adhered to and the reenactment of the Pacification of
Zaelin thoroughly reenacted, with bloodletting and prayer ensuing.

They were now sipping fresh hot uwe tea, comfortably seated at the
General's planning chamber, all sorts of charts and maps laid out
over a grand table, heaps reports and still unsgined orders heaped
upon the General's desk, a utilitarian piece of furniture, like most
around the chamber, sturdy and well-made but otherwise unadorned and
plain. 

The Archminister was seated on the only luxurious chair available,
plush velvet adorning the back and the sitting surface, elegantly
inlaid pieces of ivory, black granite and tetherwood intertwined in
flowing designs in the stylish armrests. The Procastinator Militant
sat at a simple stool, much more accomodating for a soldier in search
of a few moments of resting one's legs during brief pauses in a battle,
rather than a man of such a high office as a Procastinator Militant.
The General had briefly apologized to the Procastinator Militant for
lack of a better seating apparatus, and explained that any and all
equipment deemed to be of an extraneous nature was being dismantled
to be put into other uses now that the preparations for the campaign
demanded every last bit of usable material. And that even included
artisan chairs made of young sycamore and inlaid with ivory, granite,
and tetherwood, not unlike the last one available for seating persons
of importance, of which the Archminister seemed to make such good
use.

Not that he implied at any point, the General continued, that the
Procastinator Militant was not a person of incalculable importance,
but alas, the Archminister had seniority according to the Law of Founding,
so it was Law that essentially demanded that the Archminister be seated
in the proper way, while he would have to make do with what little
was available at such a time.

At that, the Procastinator Militant withdrew from any thought or intention
of protesting, and simply accepted the proferred stool graciously.
By looking at the Archminister, if one didn't know better he might
misinterpret his slight grin as an indication of silent enjoyment
of the unfortunate predestinations of the Procastinator on his behalf,
but such a man was beyond such base thoughts, and was merely grinning
at the studious labor going on around the Strategium Proper, praising
the high spirits of everyone involved, and personally congratulating
the General of the Army for kicking off the preparations in the way
expected:

{}``Well done, Tyrpledge. I see that you are already thinking of
using all available material. Even using the ivory and granite in
such a fine chair. Hard to find materials, very important, are they
not?'', the Archminister inquired, his nose delicately poised over
his cup of uwe, letting the aromas seep in of their own volition.

{}``I am more than honoured, indeed blessed, to hear such praise
from your Excellence. Yes, they are most valuable, as well as tetherwood
and the sycamore. From what I know of the artisan's techniques, the
ivory is used in delicate steamer parts without which the damnable
things would blow up before going ten feet. The sycamore and tetherwood
are used in constructing siege engines, and the granite is turned
into pellets for the steamers' slingshots.'', General Tyrpledge answered,
with a hint of a smile and his eyes darting back forth between Gomermont,
the Procastinator Militant, and La Vasse, to whom he added as an afterthought:

{}``Is the uwe to your liking? I can always call up the cook to present
himself and receive proper chastisement if he has failed you. He was
specifically instructed on the required quality of the uwe and your
precise likings. It would be an affront to the Council if he could
not serve properly.''

{}``There will be no need, General. Please, call me La Vasse. We
rarely meet on official business as it is with you spending most of
the time on exercises away from Pyr, and me always busy at the Ministry
and the Disciplinarium. I believe that in such an important time,
we should dispense with tiring mannerisms of protocol and etiquette,
and get on with the business in hand, to better serve Law and the
glory of the Pantheon, of course.'', the Archminister's tone polite,
level, straightforward, as if the General was his peer, which was
strictly speaking, false.

Tyrpledge was visibly but also pleasantly surprised, his look widened
and his face brightened up a tone. Gomermont seemed to fidget uncomfortably
at his stool, unable to arrange his body in a manner both sufficient
and comely of a Procastinator Militant.

La Vasse and Tyrpledge largely ignored Gomermont's discomfort and
Tyrpledge replied in kind to the Archminister:

{}``I am more than grateful for that dispensation then, La Vasse.
It does help a great deal when going to war when you don't have to
devote precious time on finding the right chair, serving the proper
tea and using the protocol-bound appelations of rank and office.'',
the General said while easing up on his chair, his body assuming a
relaxed position.

{}``Oh, make no mistake Tyrpledge, my rank, office, and related trappings
of my status as Archminister still hold and I expect you to diligently
administer the proper respect. At least in public, when we are not
planning together, exchanging information and agreeing to our next
best course of action. Be reminded of course, that the Castigator
is always briefed on our meetings and though we have been given executive
control of the Holy Campaign, whatever course of action we decide
on, has to be ratified by His Piousness. In grave matters of battle
that is, since currently I have been empowered with freedom to act
as the Castigator's proxy in these preliminary stages of the preparation.''

La Vasse's voice assumed a harsher tone, the weight in his voice and
words punctuating his heightened status of authority. His strict but
fair tone was indicative of his intentions: He would be reasonably
cooperative and would dispense of the pleasantries and honours where
applicable, but that would not bring Tyrpledge up to the same level
as him, the proxy of the Castigator. He also seemed to limit this
compensation to Tyrpledge alone, since Gomermont apart from being
a relatively useless dolt, the most common type of Procastinator,
his office also was not immediately pertinent to the Campaign, since
he and his men would remain in the cities and towns, ever watchful
of signs of insurrection and heresy when the Castigator and much of
the governing mechanism would be in the Widelands.

Tyrpledge was simply a soldier, a sword to be wielded like a tool,
bending to the master's will. He had never had any misconceptions
of his status, and the Archminister's words carried no different message:
he would still be following orders diligently and respectfully, he
just didn't have to stand at attention the whole time.

Once La Vasse's words settled in, Tyrpledge said in a simple, straightforward
manner and a genuine voice of calm acceptance:

{}``I understand perfectly, Archminister.''

Gorgemont was standing up, having given up on the stool, and he was
languidly peering over the milling mass of soldiers, artisans and
labourers outside at the huge staging fields, sipping on a freshly
poured cup of uwe. He asked noone in particular, in a rather rude
manner without turning to face either one of the men he was supposed
to be working closely with:

{}``These are the Army's infamous steamers then? They do seem clumsy
and unwieldly. They lack that polish I thought the Army insists on
fervently. And how do you fit the horses inside that? How do they
breathe, is it through those pipes? I'm quite curious.''

General Tyrpledge rolled his eyes in an almost shocking expression
of unadulterated disesteem towards Gomermont. The Archminister was
smiling, sipping almost indifferently at his uwe when the General
sighed and replied in as much seriousness as he could muster:

{}``The are called steamers, because they use steam, which is very
hot water. They do not use horses. They are not polished because if
they were, they would give away their position hours away before reaching
their intended targets. The pipes are part of the system of steamworks.''

Gomermont was adamant as he was ignorant:

{}``Ah, I see. Still, clumsy pieces of machine. I'll never understand
why you insist on using them.''

{}``I can accept that in good grace, Procastinator Militant.'',
said Tyrpledge and left the dead-end exchange of words at that. Tyrpledge
resumed his thoughts even as Gomermont took in more of the vast work,
construction, and camp area. At length, the General asked the Archminister:

{}``La Vasse, I need to know. You are closer to the Castigator, his
proxy, probably the only reliable person I can talk with meaningfully.
The Widelands are wild lands, there are no people living there. Sane
people, at least. No cities, or towns, or anything to capture, and
maintain. With no population to enlighten, no forces arrayed against
us, what objectives should I designate? What provisions will I require?
What manner of equipment, what disposition of forces? How will our
forces move? What, exactly, will we be attacking, Archminister?''

Tyrpledge's tone showed some anxiety, some words bursting forth rapidly
behind others. He was not scared, La Vasse could see that. He simply
needed a target to focus on.

{}``I am much at a loss as you are, General. I have had little foreword
of the Castigator's decision, and though privy to most of his thoughts
and discussions, I have to say that the Patriarch is better informed
than I am. All I can tell you is that you should commit the totality
of our armies, for a reconnaisance in force.''

Tyrpledge frowned in disbelief before asking to make sure, surprise
more than evident:

{}``The totality of our forces? In two weeks?''

{}``I have not been known to impart His Piousness' words imperfectly.
The sum of the armies, Tyrpledge, in two weeks.''

{}``But.. There is no precedent of such a mobilisation.. The artisans
and laborers at my disposal cannot cope with such a workload even
if I drive them to death thrice over! It is not a matter of ability,
it is simply a ..''

The general's protests were politely interrupted by the Archminister
waving a dismissive hand and saying as he reached for another cup
of uwe:

{}``The Army has been granted special dispension to use any and all
capable men and resources that can be found across the land, for the
period of time up to and including the Holy Campaign, with the blessings
of the Ministry and the cooperation of the Procastinators.''

Tyrpledge was stunned in silence and was instantly awed at the power
put forth by the Castigator, effectively forcing everyone to serve
as labour and offer his belongings for the express purposes of this
Last Holy Campaign. Truly momentous times they were living in, he
thought. And then started mentally calculating the manpower he would
need to use to have everything ready in time, when the Archminister
commented on his tea:

{}``Fine uwe, Tyrpledge. You have a fine cook. If the rest of the
army proves as capable and willing, the Pantheon will smile upon us.''

To which the Procastinator Militant added morosely:

{}``I prefer keplis to uwe, really. It upsets my stomach.''


\section{The longest errand}

The previous night's walk had exhausted him. He had laid down to sleep
right after dawn, his feet sore, his legs leaden with the weight of
all the distance traveled so far. It was indeed a long ride from the
northern lands, from Nicodemea south through the great farmlands of
Rubnis. Then crossing the river Shielwa, and onto the western rough
country of Ilonas, the shepherd country, more animal, hill and rock
than man. This was where the marble road leading into the Widelands
begun. This was where his quest had taken him so far. For weeks he
had been on the road, suffering fools too gladly sometimes, subjecting
his body into such a trial of strength of will and body as travelling
on foot for almost what seemed to be half around the world. Indeed
a feat in itself, it was simply the means to far greater a prize,
the complete knowledge of which still eluded him, despite all the
years of studies and inquiries, both his and his masters'.

The marble road started off as a narrow, thin road, small edges of
pure white marble-like material delineating its boundaries. It was
not really made from marble, for if it was it would have been demolished
and chipped away bit by bit long ago. But it seems to defy any tool
and machination of man, neither pickaxe nor chisel capable of even
slightly damaging the road. A sleek, shiny white-grey road, cold to
the touch but fine and delicate, like glasswork. But unbreakable,
unyielding, unscathed by time, man, or nature. A foreign body so exquisitely
crafted that it is indeed unique, and no artisans at any time, and
no empire that ever rose and fell ever managed to construct such a
piece of perfection, truly as some poet once said {}``for the Gods
to walk upon the lands''.

It was, and had always been, part of the lands, but alien to them
as well. The people had always known of the marble road, just as they
knew of the trees, the mountains and the rivers, the forests and the
glenns, the fields and the wheat, the goat and the cow, the suns and
the moon. But these things were of nature, and the marble road clearly
was not, for nature abhors uniqueness. Animals come in pairs, rivers
abound, so do trees. But there is only one marble road. A perfect
thing; a left over from the time Gods walked among us. Or even so,
before us.

What reason was there behind it? Why does it lead into the Widelands?
What is it made of? And who made it? With what tools, what materials
did they use? They, because this must surely be the work of thousands.
No single man could ever accomplish such a work in his lifetime. Perhaps,
most rightly so, it is the work of the Gods. And to try and unravel
their reasoning and purpose can only lead to madness, heresy, or both. 

Molo decided to leave such thoughts aside, thoughts which beget questions
that begged for answers he could not find. At least not before he
ventured into the Widelands proper, until he found what Umberth described
as the Necropolis, where inestimable knowledge was waiting to be uncovered
to the world. Knowledge of a time unknown, perhaps before man ever
walked the lands. The Time of the Gods.

It was already a fascinating sensation, walking upon the very same
road that even the Gods might have walked upon once. What other man,
apart from him and Umberth had dared walk the marble road unto its
terribly unknown end? What other man, who lived long enough to tell
the tale? What other man, who was not hunted down as a heretic, a
blasphemer? What other man who didn't have a tragic, miserable end? 

He grinned wickedly at these thoughts, for they were immediately followed
by his aspirationg: He wouldn't perish neither in the Widelands, nor
at the hands of a fanatical mob, or the ever watchful Procastinators.
He would not succumb to any torture the Ministers might want to put
him through for when all his trials and tribulation had come to pass
this, he would not be simply a man anymore. He would not be hunted
down, or exiled, or even held at bay, as a feared and terrible man.

No, when all the power and majesty and magnificence of the Gods was
unveiled and made manifest through him, he would be transformed into
a being of awe and power that the lands had not witnessed since the
beginning of time. He would become a living deity, an avatar of the
Gods, and he would be loved, and cherished, and worshipped as a God
among men should. He knew the truth of it, he could feel it in his
heart and bones, see it in his twisting dreams. Dreams of cleansing
light and fire, himself a creature of wrath and glory, terrible power
at his hands and unimaginable purpose in his mind. The purpose of
the Gods. Indeed it was their divine plan. Conceived and hatched untold
aeons ago, and he was their chosen instrument. He would not fail them.
For the lust of that power burned deep withing, deeper than the need
for breath itself.

But he had indeed to find the Necropolis first, and that task seemed
ever so slightly more difficult with each passing day. Last night
he had found the marble road, and eagerly walked under the stars for
a long stretch of time, without pause. He had seen the trees give
way to bush, the grass wither, the sounds of animals grow weaker,
fainter, fewer. He knew he was entering the Widelands, the signs visible
around him. It had been the same with Umberth, as he had recorded.

When he laid down to sleep near the marble road, under a skinny old
withered bark of a tree, a cluster of rocks sheltering him from the
winds, he put his cloak under his head as a pillow, and drew his blanket
high enough to cover his face from the rising suns, and slept lightly,
with a smile on his face, as if he was merely a content child. When
he woke up in the afternoon, the suns still high, he was more than
surprised to see that the road was not there. In fact, the road was
nowhere in sight, as if he had dreamt of how he got where he was,
or as if it was all in his mind which was starting to fade away into
chaos, and madness. 

Molo was sure he had been traveling in the correct direction. More
than sure, he felt certain it was the right direction from the beginning.
He had studied the maps his master had crafted painstakingly, with
reverent attention to detail many times over, and he was certain he
had correctly identified some, if not all of the landmarks mentioned
in Umberth's tale. So he was sure that when he stepped on the glistening
white road, hard and unyielding, though soft to the touch, almost
like metal but more like porcelain or clay, he was indeed walking
on the marble road. And that when he got off the marble road in search
of a place to sleep awhile and rest before following it once more,
it would still be there, an undisturbed reality, a known quantity,
a fact. By the Gods, it was a \textsl{road}! Not a river to overflow,
or dry up, or change its course! And even rivers have been known to
take their time in such happenstance! How can something like the marble
road disappear in a matter of hours?

Perhaps he was indeed losing his mind. Perhaps Umberth was a crazed
old fool, and his esquire doubly so. And it had all been a fantasy
to stir the minds most weak, those who were most prone to fall for
grand visions, tales of mystery and untold secrets. And he had followed
that fantasy in vain, like a fool the sort of which he despised and
felt little less than pity for. He was a fool, half-mad and soon quite
lost, left to fend for his life in this hostile land, with nothing
of worth or substance to live off it. Such an unfitting end to a journey
that should have changed the world.

Black despair seemed to take over him, his head swimming in a sea
of moroseness, thoughts of ruin and death his mind's sole occupation.
He was tense with bad temper. His fists were clenched, banging against
a rock once every so often, as if it alone was to blame for his meandering
path up to this foolishness. His despair turned into rage, overwrought
since he was with anger at his failure. He retraced his thoughts and
conversed silently with himself. 

Would he accept an ignominious defeat at the hands of fate? Would
he blindly give in to the temptation of despair? Throaw away the years
of studies and preparation? His long walk, his peregrination to the
Land of the God unfinished, nothing but an exercise in futility? And
to what purpose would he devote himself now? What other singular task
can match his ambition, his dreams beyond the realm of mortal men?

And what if he was indeed a madman? What of it? Madmen answer to noone,
only to the Gods. And so would he answer. The Land of the Gods beckoned.
They tested his mettle. Only the one who is mad enough to challenge
such authority can truly knock on the Holy Gates. He was all alone
out here, in the Widelands. The marble road disappearing, was just
the start. The beginning of the play the Gods seemed to love so much.

He would play. He would play in anything they chose to throw at him.
He would find the Necropolis, at any cost. This, was merely an inconvenience,
one of many he should still encounter. For once he set out, he knew
he would be attempting a feat that almost noone had succeeded at before.
And even if he did, as Umberth might have had, nothing was certain
of the power therein, and how it would finally become his own.

He would have to see for his own, marble road or not. He got up decisively,
and walked back to where he remembered the road lay exactly. He was
standing either right on a spot where there was a road to be seen
last night, or only a few feet away, he was certain. The grass had
a wholesome quality, as if it had always covered the same ground as
it did now, as if there never was a marble road here, not ever.

Then it was an illusion. A mirage. The only logical explanation he
could arrive to with what little he knew of the marble road, and ignoring
the possibility of him being mentally unstable, which was not helpful
if at all true, this was the only reasonable explanation. The marble
road was a lie, and by some sort of means unknown to man certainly,
it was there to be seen, only to lure those that were mentally unprepared
and easily misled and fooled deeper into the Widelands. A fly trap
of sorts, he gathered. Well, that would be of little significance
now that he had uncovered that the marble road was in fact not there
at all.

His thought took him to a passage from Umberth's tale, which made
more sense now that he had seen this happen with his very own eyes.
The passage read:

{}``Only a few nights after we had gone definetely in Wideland country,
we lost track of the road. Deciding to camp quite a distance away
from the road, towards what had seemed to be a natural spring in a
rock formation, we had lost sight of the marble road. We were doubly
misled, since the spring had dried and was no more. The maps had failed
us early on. Terlet went mad and master Umberth ordered me to put
him out of his misery. Nubir and Vamden probably got lost trying to
find the road, or simply decided to vanish before we went deeper into
the Widelands. In any case, we never saw or heard of them again.''

Molo had thought that they had indeed lost teir bearings in the difficult
to navigate terrain of the Widelands. That was why he didn't stray
out of sight of the road. But it seemed that had little effect. The
road had vanished seemingly of its own volition, or by forces and
plans he could not understand and certainly not control. At least
for now. 

He decided to continue on the same direction he had taken as before,
when the road was still visible. The exact location of the Necropolis
was a mystery, but he had arrived at the conclusion that it lay far
enough deep into the Widelands, into the Dunes of the Widelands, the
desert proper. So he would grind along purposefully with the same
eastward direction, deeper and deeper into the Widelands. He would
try and record the distances traveled, counting his steps, in order
to both focus his mind, as well as keep an account of how far deep
each day of travel takes him. There will be time to recollect, muse,
and decide on every next step when he stops to rest and take sustenance
and water, two things that will indeed be scarce.

He still had a ready supply of honey-laden bars of nuts and sesami,
a confection highly energetic, most appropriate for travelling long
distances and generally when consuming one's energy. His water sack
was still full from yesterday, but he should definetely try and keep
its usage at a minimum. The more the Widelands turned into a desert,
the more imperative it came for him to conserve his water. That is
why he had chosenn to travel at night, both to use its cover if anyone
had been following him, and certainly because it was cooler at night,
the walk not as demanding in both food or water.

But he still should be able to find some water until he went deep
in the Dunes. There, he doubted any water could be found at all for
days on end. And that was where he would either perish or triumph.
Deep in the desert dunes of the Widelands, searching for the Necropolis.

As was his preferred way, what he thought was the most sensible one,
he waited for dusk to come and the suns to come down the sky, before
he would start walking again. He picked up his knapsack with what
little more he carried within, picked up his walking stick and started
off once more, feeling he had one a small battle today, renewed vigor
and determination coursing through him. He needed no road to find
his path, he would instead tread on relentlessly, and his path and
journey would be sung in the aeons to come.

He kept a count of his steps in his head, and focused his eyes on
a specific star to follow, trying not to lose track of either his
footsteps or his direction. And as he trod on, the star firmly fixed
in his gaze, he thought he saw a great silver-white line appear in
front of him, like someone had let open open a small slit of glittering
light amidst the rocky desert night. He blinked, thinking he had not
rested long enough, that the whole incident with the road was taxing
his mind.

And then he had this rush of exhilaration and hope, and started running
towards the thin sliver of white in the dark. He ran as if his life
depended on it, forgetting everything about counting steps or going
in a straight line. He just ran ahead towards the straight white line
that seemed to fill the horizon and was coming ever so closer to him.

And when the line had grown past a line into something recognisable
indeed familiar, his aching feet and burning lungs meant nothing at
all, for he was once again stepping on the marble road. He threw his
head back and let his body drop on the road, laughing like a child
who had found a toy he had thought lost.


\part{Per Ardua}


\chapter{Wishes of the Unholy}


\section{Circumstance and happenstance}
\end{document}
